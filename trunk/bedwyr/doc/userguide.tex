\documentclass{article}

\usepackage{url}

\title{{\Huge Bedwyr} \\ User Guide}
\author{David Baelde, Andrew Gacek, Dale Miller, Gopalan Nadathur}

\begin{document}
\maketitle

\tableofcontents

\section{Introduction}

In the Section \ref{sec:install} we install Bedwyr as distributed
in tarballs or from the SVN.
In Section \ref{sec:interface} we describe the way the user interacts with 
Bedwyr, the command-line, the prompt and the syntax.
In Section \ref{sec:howto} the user should finally understand the main concepts 
involved in order to use Bedwyr efficiently.

% ============================================================================
\newpage
\section{Get Bedwyr}
\label{sec:install}

You can get Bedwyr from Slimmer's INRIA Gforge project page:
\url{http://gforge.inria.fr/projects/slimmer}.
There you can download tarballs or get the development version using SVN
| instructions are provided on the project page.

\subsection{Distribution layout}

The Bedwyr distribution is organized as follows:

\begin{tabular}{rl}
  \texttt{/src}      & Source code \\
  \texttt{/doc}      &  Documentation | you're reading it \\
  \texttt{/examples} &  Examples | reading them helps
\end{tabular}

\subsection{Build}

Bedwyr's only dependency is the OCaml compiler suite.
Then, the procedure is quite simple.

\begin{verbatim}
# ./configure
# make
\end{verbatim}

You'll get the bedwyr executable in \texttt{src/bedwyr}.

By default, Bedwyr is built using the nativecode compiler \texttt{ocamlopt},
since it is much faster. If you don't have it or don't want it (e.g.
for easier debugging) use \texttt{./configure --disable-nativecode}.

\subsection{Test}

\begin{verbatim}
Testing the core library:
# make -C src test
More tests, including examples in Bedwyr:
# make test
\end{verbatim}

% ============================================================================
\newpage
\section{User interface}
\label{sec:interface}

\subsection{Syntax}

Bedwyr uses HOAS (Higher-Order Abstract Syntax), which means that formulas
of the Linc logic and of the object logics are represented using 
$\lambda$-terms. The term normalization is implicit, the equality is the 
equality of $\lambda$-terms, notable handling implicitly $\alpha$-equivalence.

The abstraction over \verb.x. in \verb.term. is denoted by \verb.x\term. | 
which is read as $\lambda x. term$. On top of that we build 
formulas: e.g. we write \verb.pi x\ x=x.
(that is $pi\;(\lambda{}x.(=\;x\;x))$)
to represent $\forall x. x=x$.

Formulas are formed as follows:
\[\begin{array}{rclp{5cm}}
form &::=& form \texttt{,}  form & conjunction \\
     & | & form \texttt{;}  form & disjunction \\
     & | & form \texttt{=>} form & implication \\
     & | & \verb.pi x\.    form  & universal quantification over $x$ \\
     & | & \verb.sigma x\. form  & existential quantification over $x$ \\
     & | & \verb.nabla x\. form  & generic quantification over $x$ \\
     & | & atomic & definition \\
     & | & term \verb.=. term & equality \\
term &::=& id & identifiers are non-empty sequences of \verb.[A-Za-z0-9/_']. \\
     & | & term \; term+ & application \\
     & | & id \verb.\. term & abstraction \\
     & | & term \; infix \; term & infix operators are
             \verb.+., \verb.-., \verb.*., \verb.->. and \verb.<-.. \\
     & | & \verb.(. term \verb.). & \\
atomic &::=& id \; term* & a possibly empty application with a constant head \\
\end{array}\]

\subsection{Running Bedwyr}

Bedwyr has two modes: one for extending definitions and one for querying.
Definitions should be put in a file, as a set of clauses.
\[ clause ::= atomic \;\verb.:=.\; term \verb|.| \]
Such files are usually named with a \verb|.def| extension, you can find several
in the \verb.examples. directory of the Bedwyr distribution.

Queries are then asked against a given set of definitions.
Any well-formed formula is a query.

In queries, variables starting with an uppercase character (\verb.A-Z_.)
are implicitely quantified existentially, and their instantiations in solutions
are displayed.

In definitions, uppercase variables are implicitely universally quantified.

You can run Bedwyr in interactive mode by simply calling \texttt{src/bedwyr}.
You'll then get the Bedwyr prompt. In the following example we load a set of 
definitions and check a theorem about it: $\lambda x.x\;x$ has no simple type.

\begin{verbatim}
dbaelde@poum bedwyr $ src/bedwyr
[...welcome message...]
?= #include "examples/lambda.def".
?= (sigma T\ wt nil (abs x\ app x x) T) => false.
+ 1ms
Yes.
More [y] ?
+ 0ms
No more solutions.
?= pi x\ X x = x x.
+ 0ms
Solution found:
 X = (x1\x1 x1)
More [y] ?
+ 0ms
No more solutions.
\end{verbatim}

A simpler way to load definition files is to pass them directly on the
command-line (call \verb|src/bedwyr examples/lambda.def|).
Other options are detailed in \verb.src/bedwyr --help..

% ============================================================================
\newpage
\section{How to use Bedwyr ?}
\label{sec:howto}

A better undestanding of the tool is probably needed to get your work done.

\subsection{Proof-search within Level 0/1}

Bedwyr is a proof-search engine for the Level 0/1 fragment of the Linc logic.
Roughly, the Level 0/1 fragment requires that the formulas on the left of
implications do not contain implications and universal quantifications.
It allows the proof-search to be simple and (more) efficient,
since the formulas on the left can be immediately eliminated.

To prove an implication $A\Rightarrow B$,
Bedwyr enumerates all possible solutions $\theta_i$ for $A$,
and then tries to prove $B\theta_1\wedge\dots\wedge B\theta_N$.
In particular when $A$ has no solution, the formula is true.

Two runtime errors can occur:
when the prover encounters an unification which isn't a higher-order pattern;
when a logic variable appears on the left of an implication.

Examples will help to understand the second one:
\begin{verbatim}
?= X=1 => X=1.
Unknown error: Failure("logic variable on the left")
?= X=1, pi x\ x = X => x = X.
+ 0ms
Solution found:
 X = 1
More [y] ?
+ 0ms
No more solutions.
\end{verbatim}
The first query requires radically different tools than those we use |
namely, disunification. The query actually has infinitely many solutions,
any term other than $1$ fits $X$.
In the second case, it works, since \verb.X. is instantiated at the time the
implication is processed.

\subsection{Tabling}

Until now, all definitions (inductive, coinductive or non-recursive)
are treated the same way, loops can occur in the proof-search, and Bedwyr
won't avoid them. Also, Bedwyr can do the same search several times without
noticing. To solve that, we use tabling.

Although quite weak, tabling in Bedwyr is a great improvement.
It is still under work.

To enable tabling for some predicate \texttt{p}, simply use the directive
\texttt{\#table p.}, in a definition file or from the prompt.
For tabled definitions, being inductive of coinductive matters.
You should thus declare it by prepending the \texttt{inductive} or
\texttt{coinductive} keyword to every clause of the definition.

For now, tabling only applies to goals which do not contain logic variables.
Then, loops are successes for coinductive definitions,
and failures for inductive ones.

\end{document}
