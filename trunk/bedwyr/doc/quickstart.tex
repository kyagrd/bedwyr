\documentclass[a4paper,twocolumn]{article}
% convenient
\usepackage[utf8]{inputenc}

% OT1 font encoding is ancient and unsuitable for special chars
\usepackage[T1]{fontenc}

% amsmath must be included before txfonts
\usepackage{amsmath} % text in math, operators, bold symbols, etc
%\usepackage{amsthm} % theorems
%\usepackage{amssymb} %fonts and more

% Type 1 fonts (vectorial) instead of Type 3 (bitmap)
%\usepackage{lmodern} % Computer Modern
\usepackage{txfonts} % Adobe Times

% wrapper around xy (preferred since understood by hevea)
\usepackage[all]{xypic}

% all sorts of links: internal, external, multi-line...
\usepackage[%
% linkbordercolor={1 0 0}, % default: red
% citebordercolor={0 1 0}, % default: green
% filebordercolor={1 0 1}, % default: magenta
% urlbordercolor={0 1 1},  % default: cyan
]{hyperref}
\usepackage{breakurl}

% hevea commands, to be ignored by latex
\usepackage{hevea}
\loadcssfile{http://slimmer.gforge.inria.fr/bedwyr/bd.css}

% MATHs PARagraphs and Type Inference Rules
\usepackage{mathpartir}

% source code highlighting
\usepackage{color}
\definecolor{commentcolour}{rgb}{0.5,0.5,0.5}
\definecolor{stringcolour}{rgb}{0.58,0,0.82}
\definecolor{keywordcolour}{rgb}{.3,.1,.1}
\definecolor{constantcolour}{rgb}{.1,.1,.3}
\definecolor{variablecolour}{rgb}{.3,.5,.9}
\definecolor{punctiationcolour}{rgb}{.3,.3,0}
\definecolor{red}{rgb}{1,0,0}
\definecolor{green}{rgb}{0,1,0}
\definecolor{blue}{rgb}{0,0,1}
\definecolor{pathcolour}{rgb}{1,.9,.8}
% source code syntax
\usepackage{listings}
\lstset{%
  frame=lines,%
  emptylines=1,%
  captionpos=t,%
  columns=fullflexible,%
}
\usepackage{include/tex/languages}
\lstdefinestyle{lprolog}{%
  language=lprolog,%
  columns=fixed,%
  basicstyle=\ttfamily\small\color{black},%
  commentstyle=\color{commentcolour},,%
  stringstyle=\color{stringcolour},%
  identifierstyle=\idstyle,%
  keywordstyle={\color{punctiationcolour}},%
  keywordstyle={[2]\color{constantcolour}\bfseries},%
  keywordstyle={[3]\color{keywordcolour}\underline},%
  keepspaces,%
}
\lstdefinestyle{bedwyr}{%
  language=bedwyr,%
  columns=fixed,%
  basicstyle=\ttfamily\small\color{black},%
  commentstyle=\color{commentcolour},%
  stringstyle=\color{stringcolour},%
  identifierstyle=\color{black},%
  identifierstyle=\idstyle,%
  keywordstyle={\color{punctiationcolour}},%
  keywordstyle={[2]\color{constantcolour}\bfseries},%
  keywordstyle={[3]\color{keywordcolour}\underline},%
  keywordstyle={[4]},%
  directivestyle={\color{keywordcolour}},%
  literate={\\\$}{{\$}}1,%
  keepspaces,%
}
\lstdefinestyle{bedwyr-repl}{%
  style=bedwyr,%
  identifierstyle=,%
  keywordstyle={[4]\sffamily\bfseries\upshape\color{black}},%
}

\bibliographystyle{alpha}

% macros
\newcommand{\lp}{$\lambda$Prolog}
\newcommand{\pc}{$\pi$-calculus}
\newcommand{\lc}{$\lambda$-calculus}
\newcommand{\Ll}{$L_\lambda$}
\newcommand{\muDef}{\overset{\mu}{=}}
\newcommand{\nuDef}{\overset{\nu}{=}}
%\newcommand{\deDef}{\overset{\Delta}{=}}
\newcommand{\qs}{\, .\,}
\providecommand{\path}[1]{\colorbox{pathcolour}{\ttfamily#1}}
\providecommand{\autoref}[1]{\ref{#1}}

\sloppy

\usepackage{amssymb}

\hypersetup{pdftitle={Bedwyr 1.4 QuickStart}}
\lstset{style=bedwyr}

\title{{\Huge A Quick-Start Guide to\\Bedwyr v1.4}}
\author{Quentin Heath\\
        Inria Saclay--\^Ile-de-France \& LIX, \'Ecole polytechnique}

\begin{document}
\maketitle

% ============================================================================
\section{Overview}

Bedwyr is a generalization of logic programming that allows model
checking directly on syntactic expression.

The logic manipulated, a subset of LINC (for \emph{lambda, induction,
nabla, co-induction}), contains object-level $\lambda$-abstractions,
$\nabla$ (nominal) quantification, and (co-)inductive definitions.
While LINC is a extension of higher-order intuitionistic logic, Bedwyr
formulae are restricted to a fragment where connectives on the left of
an implication must have invertible rules (i.e. no universal
quantifier nor implication -- this enables the closed-world assumption
when identifying finite success and finite failure), while equalities
are restricted to the \Ll{} fragment (allowing for the use of
higher-order pattern unification).

The OCaml implementation is part of an open source project.  The
\href{http://slimmer.gforge.inria.fr/bedwyr/#download}{web page}
offers multiple ways to get it:
\begin{itemize}%
%BEGIN LATEX
[nolistsep]
%END LATEX
  \item SVN repository
  \item tarballs
  \item precompiled binaries
  \item GNU/Linux packages
  \item Windows installer
\end{itemize}

The
\href{http://slimmer.gforge.inria.fr/bedwyr/#documentation}{documentation}
includes this quick-start guide, a reference manual and the
source-code documentation.

% ============================================================================
\section{Input}

Input sources can be plain text definition files (\path{*.def}), the
REPL (read-eval-print loop), the \lstinline+read+ predicate, and
arguments to the \texttt{-d}, \texttt{-e} and \texttt{-c} CLI
(command-line interface) options.  Three modes are available:
\begin{description}%
%BEGIN LATEX
[nolistsep]
%END LATEX
  \item[definition mode] is used in \path{*.def} files and with
    \texttt{-d}
  \item[toplevel mode] is used in the REPL and with \texttt{-e}
  \item[term mode] is used at run-time via \lstinline+read+ and with
    \texttt{-c}
\end{description}

\begin{figure}[t]
  \centering
  \begin{tabular}{l|cccc}
                                                         & *.def      & REPL       & \lstinline+read+ & CLI \\
    \hline
         \ref{sec:commands} \lstinline+Define p : prop.+ & \checkmark &            &                  & \texttt{-d} \\
          \ref{sec:queries} \lstinline+X = 42.+          &            & \checkmark &                  & \texttt{-e} \\
    \ref{sec:meta-commands} \lstinline+#env.+            & \checkmark & \checkmark &                  & \texttt{-d/-e} \\
            \ref{sec:terms} \lstinline+1 :: 2 :: nil.+   &            &            & \checkmark       & \texttt{-c}
  \end{tabular}
\end{figure}

\subsection{Commands}
\label{sec:commands}

Commands are used in definition mode to declare new types and
constants, declare and define predicates, and define theorems (which
are not proved, but are used as lemmas).

\subsection{Queries}
\label{sec:queries}

Queries are plain formulae (terms of type \lstinline+prop+), entered
at the toplevel, that Bedwyr attempts to solve.  If possible, it
outputs the list of solutions (substitutions of the free variables).
Otherwise, if the formula is not handled by the prover (non-invertible
connective on the left) or by the unifier (not \Ll{}), resolution
aborts.

\subsection{Meta-commands}
\label{sec:meta-commands}

Meta-commands are used both in definition mode and at the toplevel,
mostly to improve the user experience by executing strictly
non-logical tasks, such as input (\lstinline+#include "inc.def".+),
output (\lstinline+#typeof X Y :: nil.+) or testing
(\lstinline+#assert true.+).  A few of them change proof structure,
but not provability (\lstinline+#freezing 4.+, \lstinline+#saturation 2.+).

\subsection{Terms}
\label{sec:terms}

Term-mode is a way to improve interactivity.

% ============================================================================
\section{Sample definition file}

\autoref{def:maxa} shows a complete sample definition file,
with the declarations for a type and two constants, along with a few predicates.

\begin{figure}[t]
  \lstinputlisting[%
    caption={\protect\path{maxa.def}},%
    label=def:maxa%
  ]{include/linc/maxa.def}
\end{figure}

The predicate \lstinline+a+ is a typical example of what must be done to build
a Bedwyr example: even with a theoretically infinite search space
(here, Church numerals), Bedwyr only does finite reasoning, and hence must be
given an explicit description of its finite actual search space.

The use of the \lstinline+inductive+ keyword has two consequences.
Firstly, memoization is used on the corresponding predicate;
secondly, it has an impact on the way loops in computation are handled.
Since the \lstinline+leq+ predicate obviously cannot loop,
only the first aspect is used here (meaning we might as well have used the
\lstinline+coinductive+ keyword instead).

% ============================================================================
\section{REPL demo}

\autoref{repl:maxa} shows an example of use of the interactive toplevel
following the invocation of \verb+bedwyr maxa.def+.

\begin{figure}[t]\begin{lstlisting}[%
  style=bedwyr-repl,caption={Interactive session},label=repl:maxa]
?= #env.
*** Types ***
   option : * -> *
   list : * -> *
   ch : *
*** Constants ***
   (::) : A -> list A -> list A
   nil : list A
   opnone : option A
   opsome : A -> option A
   s : ch -> ch
   z : ch
*** Predicates ***
   a : ch -> prop
 I leq : ch -> ch -> prop
   maxa : ch -> prop
   member : A -> list A -> prop
?= read Y /\ leq X Y.
 ?> s z.
Solution found:
 X = z
 Y = s z
More [y] ? y
Solution found:
 X = s z
 Y = s z
More [y] ? y
No more solutions.
?= maxa X.
Solution found:
 X = s (s (s (s (s z))))
More [y] ? y
No more solutions.
?= #exit.
\end{lstlisting}\end{figure}

The \lstinline+#env.+ meta-command shows all declared objects
(including the standard pre-declared list-related objects), and
informs \lstinline+leq+ is inductive.  The call to the queries
\lstinline+leq X (s z).+ and \lstinline+maxa X.+ offer to display all
solutions, one by one.
A subsequent call to \lstinline+#show_table leq.+ would show the table
filled by the second query with \lstinline+leq+-headed atoms, either
marked as proved or disproved.

\end{document}
