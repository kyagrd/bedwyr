\part{A User's Guide to Bedwyr}


% ============================================================================
% ============================================================================
\section{Overview}

Some recent theoretical work in proof search has illustrated that it is
possible to combine the following two computational principles into one
computational logic:
\begin{enumerate}
  \item a symmetric treatment of finite success and finite failure --
    this allows capturing both aspects of \emph{may} and \emph{must}
    behavior in operational semantics, and mixing model checking and
    logic programming;

  \item direct support for $\lambda$-tree syntax, as in \lp{}, via
    term-level $\lambda$-binders, higher-order pattern unification, and
    the $\nabla$ quantifier.
\end{enumerate}
All these features have a clean proof theory.  The combination of these
features allow, for example, specifying rather declarative approaches to
model checking syntactic expressions containing bindings.  The Bedwyr
system is intended as an implementation of these computational logic
principles.

\paragraph{Why the name Bedwyr?}
In the legend of King Arthur and the round table, several knights shared
in the search for the holy grail.  The name of one of them, Parsifal, is
used for an INRIA team where Bedwyr is currently developed. Bedwyr was
another one of those knights.  Wikipedia (using the spelling
``Bedivere'') mentions that Bedwyr appears in \emph{Monty Python and the
Holy Grail} where he is ``portrayed as a master of the extremely odd
logic of ancient times''.  Bedwyr is a re-implementation and rethinking
of an earlier system called Level 0/1 written by Alwen Tiu and described
in \cite{tiu05eshol}. It was an initial offering from ``Slimmer'', a
jointly funded effort between INRIA and the University of Minnesota on
``Sophisticated logic implementations for modeling and mechanical
reasoning'' from 2005 to 2010. For more information, see
\url{http://slimmer.gforge.inria.fr/}.

\paragraph{What is the difference between \emph{hoas} and
  \emph{$\lambda$-tree syntax}?}
The term ``higher-order abstract syntax'' (hoas) was originally coined
by Pfenning and Elliott in \cite{pfenning88pldi} and names the general
practice (that was common then in, say, \lp{} \cite{miller87slp}) of
using an abstraction in a programming or specification language to
encode binders in an object-language.  Since the choice of
``meta-language'' can vary a great deal, the term ``hoas'' has come to
mean different things to different people.  When hoas is used directly
within functional programming or constructive type systems, syntax with
bindings contains functional objects, which make rich syntactic
manipulations difficult.  Bedwyr, on the other hand, follows the
\emph{$\lambda$-tree} approach \cite{miller00cl} to hoas: in particular,
Bedwyr's use of $\lambda$-abstraction is meant to provide an abstract
form of syntax in which only the names of bindings are hidden: the rest
of the structure of syntactic expressions remains.

\paragraph{Is Bedwyr efficient?}
Some care has been taken to implement the novel logical principles that
appear in Bedwyr.  In particular, the system makes extensive use of the
implementation of the suspension calculus \cite{nadathur99jflp} and
other implementation ideas developed within the Teyjus
\cite{nadathur99cade} implementation of \lp{} \cite{nadathur88iclp}.
Aspects of tabled deduction have also been added to the system
\cite{ramakrishna97cav,pientka05cade}.  We have found that Bedwyr's
performance is good enough to explore a number of interesting examples.
It is not likely, however, that the current implementation will support
large examples.  For example, the system implements the occur-check
within logic: this is, of course, necessary for sound deduction but it
does slow unification a lot.  As a result, the append program is
quadratic in the size of its first argument.  There are a number of
well-known improvements to unification that make it possible to remove
many instances of the occur-check (and making append linear).  As of
this time, such an improvement has not been added to Bedwyr.

\paragraph{An open source effort: Can I help?}
The Bedwyr system was conceived as a prototype that could help validate
certain proof theory and proof search topic.  In the end, this prototype
has illustrated the main principles that we hoped that it would.  It has
also pointed out a number of new topics to be explored.  If you are
interested in contributing examples, features, or performance
enhancements, or if you are interested in considering the next
generation of a system like this, please let an author of this guide
know: we are looking for contributions.

\paragraph{Background assumed}
To read this guide, we shall assume that the reader is familiar with the
implementation of proof search that is found in, say, Prolog, \lp{}, or
Twelf.  While familiarity with various foundations-oriented papers
(particularly,
\cite{mcdowell03tcs,miller05tocl,tiu04phd,baelde08lfmtp,tiu10tocl}) is
important for understanding fully this system, much can be learned from
studying the examples provided in the distribution.


% ============================================================================
% ============================================================================
\section{Getting Bedwyr}

Different means of getting Bedwyr are listed on Slimmer's INRIA Gforge
project site: \url{http://slimmer.gforge.inria.fr/bedwyr/#download}.
You can either download tarballs, get any development version using SVN,
or use Slimmer's unofficial Apt repository -- instructions are provided
on the project page.  The development of Bedwyr is meant to be an open
source project.  If you are keen to work on the source code and/or
examples, please contact one of the ``Project Admins'' of the project
(as listed at \url{https://gforge.inria.fr/projects/slimmer/}).

% ============================================================================
\subsection{Distribution layout}

The Bedwyr distribution is organized as follows:

\begin{tabular}{r@{\quad}l}
  \path{src/}      & Source code \\
  \path{doc/}      & Documentation -- you're reading it \\
  \path{contrib/}  & Auxiliary files -- e.g. Emacs and Vim support \\
  \path{examples/} & Examples -- reading them helps
\end{tabular}

% ============================================================================
\subsection{Build}

Bedwyr's main build dependency is on the OCaml compiler suite.  You also
need some standard tools you probably already have, especially the GNU
build system (a.k.a. the autotools, especially autoconf and GNU make),
bash, tar, gzip, bzip2, etc.  Most of these are looked for by the
configuration step.

Then, the procedure is quite simple:

\begin{verbatim}
 $ make
\end{verbatim}

You'll get a link to the Bedwyr executable as \path{./bedwyr}.  You can
also use the expanded form to choose your building options:

\begin{verbatim}
 $ autoconf
 $ ./configure <configure-options>
 $ make <make-targets>
\end{verbatim}

The available options and targets are listed with

\begin{verbatim}
 $ ./configure --help
 $ make help
\end{verbatim}

By default, Bedwyr is built using the bytecode compiler \verb.ocamlc.,
since compilation with it is much faster.  If you don't want this (no
frequent recompilation, no need for debugging), you can generate more
efficient native-code instead with \verb.ocamlopt. by passing the option
\verb.--enable-nativecode. to \verb|./configure|.

You can also enable the documentation generation by using
\verb|--enable-doc| and then \verb|make doc|.  This userguide, the
quick-start guide the \verb|ocamldoc| documentation will be generated
in \path{doc/}.

Last, \verb|--with-xmlm|/\verb|--without-xmlm| will add/remove a
dependency on the \verb|xmlm| OCaml library, needed to produce XML
output.  By default, the library is not required, and the feature is
included if and only if the library is found.

% ============================================================================
\subsection{Test}

Individual components can be tested for bugs with individual targets:

\begin{verbatim}
 $ make test_ndcore
 $ make test_batyping
 $ make test_input
 $ make test_prover
\end{verbatim}

This longer test runs the complete program on a set of example files:

\begin{verbatim}
 $ make test_bedwyr
\end{verbatim}

The \verb.test. target runs all the previous tests, and serves as a
correctness and performance test.


% ============================================================================
% ============================================================================
\section{User interface}

When you run Bedwyr, you specify a list of definition files, which
contain objects to be declared and defined.  You can then use the
toplevel to ask queries against those definitions.

There is also a special brand of commands, meta-commands, which can
appear anywhere.

As a general rule, any kind of input ends with a full-stop. Commands
start with uppercase letters, meta-commands with hashes, and queries
are just formulae.

% ============================================================================
\subsection{Definition files}

Definition files are usually named with a \verb|.def| extension.  You
can find several of them in the \path{examples/} directory of the Bedwyr
distribution.  They contain declarations for types
(\lstinline{Kind <id> type.}), declarations for constants
(\lstinline{Type <id> type.}), declarations and definitions for
predicates (\lstinline{Define <id> : <type> by <definitions>.}), and
theorems (\lstinline{Theorem <id> : <formula>.}).

The only meta-command that is really intended for definition files is
the include command: \lstinline{#include "another/file.def"}.  This can
really be seen as the plain inclusion of another file, as Bedwyr doesn't
have any namespace or module system.  If the path is not absolute, it is
relative to the path of the current file, or the execution path for the
toplevel.

\subsubsection{Emacs mode}

Assuming Bedwyr is installed in standard Linux system folders, you can
use the Emacs mode for Bedwyr by adding these two lines to your
\path{~/.emacs} file:
\begin{verbatim}
(load "/usr/share/bedwyr/contrib/emacs/bedwyr.el")
(setq bedwyr-program "/usr/bin/bedwyr")
;; Of course you can change both locations to wherever you want.
\end{verbatim}

Then you should be able to load any \verb:.def: file and have syntax
highlighting and some rough auto-indenting.  Also if you do
\verb.C-c C-c. it will start Bedwyr and load the current file you are
working on.

\subsubsection{Vim syntax highlighting}

There is also a basic syntax highlighting file for Vim. With a standard
system installation, the files
\path{/usr/share/vim/vimfiles/[ftdetect|syntax]/bedwyr.vim} should
suffice; otherwise do the following:
\begin{itemize}
  \item copy \path{contrib/vim/syntax/bedwyr.vim} to your
    \path{~/.vim/syntax/} to make highlighting available
  \item copy \path{contrib/vim/ftdetect/bedwyr.vim} to your
    \path{~/.vim/ftdetect/} to have it used automatically for all
    \verb|.def| files
\end{itemize}

% ============================================================================
\subsection{Toplevel}

The interactive toplevel is automatically launched once the files have
been parsed, unless the flag \verb.-I. is passed to Bedwyr.  In it, you
can either query a formula, or run a meta-command.  In queries, free and
bound variables are the only objects that can be used without prior
declaration, and the solutions are displayed as instantiations of free
variables.

Queries can also be given in batch mode, to a non-interactive toplevel,
via the command-line option \verb.-e. (e.g.  \verb|bedwyr -e 'X = 0.'|).
In this case, they are processed after the files and before the
interactive toplevel.

In \autoref{bd:lambda-run} we load a set of definitions and prove
(twice) that the untyped $\lambda$-term $\lambda x.x\;x$ has no simple
type.

\begin{figure}
  \centering
  \begin{lstlisting}[%
    style=bedwyr-repl,%
    caption={Run on \protect\path{examples/lambda.def}.},%
    label=bd:lambda-run]
?= (exists T, wt void (abs x\ app x x) T).
No.
?= (exists T, wt void (abs x\ app x x) T) -> false.
Yes.
More [y] ?
No more solutions.
?=
  \end{lstlisting}
\end{figure}

Notice that we had to use the term \lstinline{(abs x\ app x x)} instead
of \lstinline{(x1\ x1 x1)}: the former encodes the untyped
$\lambda$-term $\lambda x.x\;x$ by mapping object-level abstraction to
\lstinline{abs} and object-level application to \lstinline{app}, while
the latter would map them directly to logic-level abstraction and
application, and therefore is not a legal term in Bedwyr.  (Prior to
version 1.3, this was allowed as Bedwyr did not use simple typing on its
own terms.)

Most of the errors that can stop the reading of a file (parsing or
typing error, undeclared object, etc.) are correctly caught by the
toplevel, though the line number reported for the error is often not
useful.

\subsubsection{Line editing}

Bedwyr has no line editing facilities at all. Thus, we recommend using
\verb|ledit| or \verb|rlwrap|, which provides such features. Get one
of them from your usual package manager or at
\url{http://pauillac.inria.fr/~ddr/ledit/} or
\url{http://utopia.knoware.nl/~hlub/rlwrap/#rlwrap}.

Then you can simply run \verb.ledit bedwyr.. One can also define an
alias in his \path{~/.bashrc}, such as the following which also makes use
of \path{~/.bedwyr_history} to remember history from one session to
another:
\begin{verbatim}
alias bedwyr="ledit -h ~/.bedwyr_history -x /path/to/bedwyr"
\end{verbatim}

% ============================================================================
\subsection{Meta-commands}

Meta-commands are used to change the state or the program, or for
non-logical tasks.  They can be used any time a command or query is
allowed.

\subsubsection{Session management}

Those commands alter the set of definitions the current session of
Bedwyr holds.  An empty session actually means that only pervasive
types, constants and predicates are known.  The session's initial state
is the list of files given on the command-line, and it can grow anytime
with the use of \lstinline{#include}.  It should be noted that, although
Bedwyr has no solid notion of what a module is, it tries to do the smart
thing by ignoring the request to include a file if it appears to be
already loaded in the current session, as failure to do this would
result in fatal multiple declarations.  This only works if the same path
is used for each inclusion; for instance, \path{./file.def} and
\path{file.def} will be seen as different files.
\begin{itemize}
  \item \lstinline{#include} adds a \verb|.def| file to the current
    session.  It is designed to be used in definition files.

  \item \lstinline{#session} is an advanced \lstinline{#include} meant
    for query mode.  It accepts any number of filenames as parameters,
    and this set of files will be remembered as the new session.  When
    you pass filenames on Bedwyr's command line, it is equivalent to a
    call to \lstinline{#session} with these definition files.

  \item \lstinline{#reload} clears all the definitions, and then reloads
    the session's initial state, i.e. the definition files given on the
    command-line.  It is useful if they have been changed.

  \item \lstinline{#reset} clears all the definitions and empties the
    session.  It is synonymous to \lstinline{#session} with no
    arguments.
\end{itemize}

\subsubsection{Assertions}

Three kinds of assertions can be used in definition files.  These tests
are not executed unless the \verb.-t. flag has been passed on Bedwyr's
command-line, in which case any assertion failure is fatal.
\begin{itemize}
  \item \lstinline{#assert} checks that a formula has at least one
    solution.

  \item \lstinline{#assert_not} checks that a formula has no solution.

  \item \lstinline{#assert_raise} checks that the proof-search for a
    formula triggers a runtime error.
\end{itemize}

Our examples include a lot of assertions, to make sure that definitions
have (and keep) the intended meaning.  These assertions are also the
basis of Bedwyr's correctness and performance tests ran using
\verb.make test..

\subsubsection{Other commands}

\begin{itemize}
  \item Output
    \begin{itemize}
      \item \lstinline{#env} displays the current session (types,
        constants and predicates).

      \item \lstinline{#typeof} displays the (monomorphic) type of a
        formula and of its free variables.

      \item \lstinline{#show_def} displays the definition of a
        predicate.

      \item \lstinline{#show_table} displays the content of the table of
        an inductive of co-inductive predicate.

      \item \lstinline{#save_table} writes the table of an inductive of
        co-inductive predicate in a fresh definition file.

      \item \lstinline{#export} exports a structured aggregate of all
        tables in a file. This functionality has to be enabled at
        compile-time.
    \end{itemize}

  \item Tabling (i.e. memoization or caching)
    \begin{itemize}
      \item \lstinline{#equivariant} enables an alternative tabling
        mode.

      \item \lstinline{#clear_table} clears the results cached for a
        predicate.

      \item \lstinline{#clear_tables} clears all cached results.

      \item \lstinline{#freezing} sets the depth of backward-chaining
        during proof-search.

      \item \lstinline{#saturation} sets the depth of forward-chaining
        during proof-search.
    \end{itemize}

  \item General purpose
    \begin{itemize}
      \item \lstinline{#debug} adds a lot of output during the proof
        search.

      \item \lstinline{#time} displays computation times between two
        results.

      \item \lstinline{#help} is what you should type first.

      \item \lstinline{#exit} is what you should type last.
    \end{itemize}
\end{itemize}
