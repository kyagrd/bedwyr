% Add some mention of
%   finite failure is incorporated.  Seldom used in the definition of
%   operational semantics (mismatch in pi-calculus is an exception).
%   Meta-level properties of SOS often are related to negation.  One
%   might want to proof the no trace reveals a secret (say, in the
%   spi-calculus).  Proving (bisim P 0) means that P makes no
%   transitions.  Etc.  Properties of SOS thus, often, involve
%   negative information.

\documentclass{article}
\bibliographystyle{alpha}
\usepackage{url}
\usepackage[all]{xy}
\usepackage{amsmath}
\usepackage{hevea}
\loadcssfile{http://slimmer.gforge.inria.fr/bedwyr/bd.css}

% XXX unfortunately, hevea doesn't seem to handle
% hyperref correctly (ie it breaks all formulas)
%\usepackage{hyperref}
\providecommand{\texorpdfstring}[2]{#1}

\usepackage{fancyvrb}
\newcommand{\mytilde}{\raise.3ex\hbox{$\scriptstyle\sim$}}
\RecustomVerbatimEnvironment{Verbatim}{Verbatim}{commandchars=\\\{\}}

\newcommand{\lp}{$\lambda$Prolog}
\newcommand{\Ll}{$L_\lambda$}
\newcommand{\qs}{\; . \;}

\title{{\Huge A User Guide to Bedwyr v1.3}
   \thanks{Support has been obtained for this work from the following
           sources: from INRIA (first through the ``\'Equipes
           Associ{\'e}es'' Slimmer, and the ``ADT'' BATT),
           and from the NSF Grants CCR-0429572 (that also included
           support for Slimmer) and OISE-0553462.}
}
\author{David Baelde$^1$, 
        Andrew Gacek$^2$, 
        Quentin Heath$^3$, \\  
        Dale Miller$^3$,
        Gopalan Nadathur$^4$, 
        Alwen Tiu$^5$ \\ \\
$^1$ITU, Copenhagen\\
$^2$Rockwell Collins\\
$^3$INRIA Saclay and LIX/\'Ecole Polytechnique\\
$^4$University of Minnesota\\
$^5$Australian National University
}

\begin{document}
\maketitle
\tableofcontents
\newpage

\section{Overview}

Some recent theoretical work in proof search has illustrated that it
is possible to combine the following two computational principles into
one computational logic:
\begin{enumerate}
  \item a symmetric treatment of finite success and finite failure.
    This allows capturing both aspects of may and must behavior in
    operational semantics and mixing model checking and logic programming.

  \item direct support for $\lambda$-tree syntax, as in \lp{},
    via term-level $\lambda$-binders, higher-order pattern
    unification, and the $\nabla$-quantifier.
\end{enumerate}
All these features have a clean proof theory.  The combination of
these features allow, for example, specifying rather declarative
approaches to model checking syntactic expressions containing
bindings.  The Bedwyr system is intended as an implementation of these
computational logic principles.

\paragraph{Why the name Bedwyr?}
In the legend of King Arthur and the round table, several knights
shared in the search for the holy grail.  The name of one of them,
Parsifal, is used for an INRIA team where Bedwyr is currently
developed. Bedwyr was another one of those knights.  Wikipedia (using
the spelling ``Bedivere'') mentions that Bedwyr appears in {\em Monty
Python and the Holy Grail} where he is ``portrayed as a master of the
extremely odd logic in the ancient times, whom occasionally blunders."
Bedwyr is a re-implementation and rethinking of an earlier system
called Level 0/1 written by Alwen Tiu and described in
\cite{tiu05eshol}. It was an initial offering from ``Slimmer'', a
jointly funded effort between INRIA and the University of Minnesota on
``Sophisticated logic implementations for modeling and mechanical
reasoning'' from 2005 to 2010. For more information, see
\urldef{\thisurl}\url{http://slimmer.gforge.inria.fr/}
\ahrefurl{\thisurl}.

\paragraph{What is the difference between {\em hoas} and $\lambda${\em
    -tree syntax}?}
The term ``higher-order abstract syntax'' (hoas) was originally coined
by Pfenning and Elliott in \cite{pfenning88pldi} and names the general
practice (that was common then in, say, \lp{}
\cite{miller87slp}) of using an abstraction in a programming or
specification language to encode binders in an object-language.  Since
the choice of ``meta-language'' can vary a great deal, the term
``hoas'' has come to mean different things to different people.  When
hoas is used directly within functional programming or constructive
type systems, syntax with bindings contains functional objects, which
make rich syntactic manipulations difficult.  Bedwyr, on the other
hand, follows the {\em $\lambda$-tree} approach \cite{miller00cl} to
hoas: in particular, Bedwyr's use of $\lambda$-abstraction is meant to
provide an abstract form in which only the names used for binding
names are hidden: the rest of the structure of syntactic expressions
remains.

\paragraph{Is Bedwyr efficient?}
Some care has been taken to implement the novel logical principles
that appear in Bedwyr.  In particular, the system makes extensive use
of the implementation of the suspension calculus \cite{nadathur99jflp}
and other implementation ideas developed within the Teyjus
\cite{nadathur99cade} implementation of \lp{} \cite{nadathur88iclp}.
Aspects of tabled deduction have also been added to the system
\cite{ramakrishna97cav,pientka05cade}.  We have found that Bedwyr's
performance is good enough to explore a number of interesting
examples.  It is not likely, however, that the current implementation
will support large examples.  For example, the system implements the
occur-check within logic: this is, of course, necessary for sound
deduction but it does slow unification a lot.  For example, the append
program is quadratic in the size of its first argument.  There are a
number of well-known improvements to unification that make it possible
to remove many instances of the occur-check (and making append
linear).  As of this time, such an improvement has not been added to
Bedwyr.

\paragraph{An open source effort: Can I help?}
The Bedwyr system was conceived as a prototype that could help
validate certain proof theory and proof search topic.  In the end,
this prototype has illustrated the main principles that we hoped that
it would.  It has also pointed out a number of new topics to be
explored.  If you are interested in contributing examples, features,
or performance enhancements, or if you are interested in considering
the next generation of a system like this, please let an author of
this guide know: we are looking for contributions.

\paragraph{Background assumed}
To read this guide, we shall assume that the reader is familiar with
the implementation of proof search that is found in, say, Prolog, \lp{},
or Twelf.  While familiarity with various foundations-oriented papers
(particularly, \cite{mcdowell03tcs, miller05tocl,tiu04phd}) is
important for understanding fully this system, much can be learned
from studying the examples provided in the distribution.


% ============================================================================
\section{Get Bedwyr}

Different means of getting Bedwyr are listed on Slimmer's INRIA Gforge
project site:
\urldef{\thisurl}\url{http://slimmer.gforge.inria.fr/bedwyr/#download}
\ahrefurl{\thisurl}.
You can either download tarballs, get any development version using SVN,
or use Slimmer's unofficial Apt repository | instructions are provided
on the project page.
The development of Bedwyr is meant to be an open source project.
If you are keen to work on the source code and/or examples, please contact
one of the ``Project Admins'' of the project (as listed at
\urldef{\thisurl}\url{https://gforge.inria.fr/projects/slimmer/}
\ahrefurl{\thisurl}.

\subsection{Distribution layout}

The Bedwyr distribution is organized as follows:

\begin{tabular}{r@{\quad}l}
  \texttt{src/}      & Source code \\
  \texttt{doc/}      & Documentation | you're reading it \\
  \texttt{contrib/}  & Emacs and Vim support \\
  \texttt{examples/} & Examples | reading them helps
\end{tabular}

\subsection{Build}

Bedwyr's main build dependency is the OCaml compiler suite.
You also need some standard tools you may already have, especially
autoconf and GNU make (part of the GNU toolchain),
and bash, tar, gzip and bzip2 for the installation.

Then, the procedure is quite simple.

\begin{verbatim}
 $ autoconf
 $ ./configure
 $ make
\end{verbatim}

You'll get the bedwyr executable in \texttt{src/bedwyr}.

By default, Bedwyr is built using the native-code compiler \texttt{ocamlopt},
since it is much faster. If you don't have it or don't want it (e.g.
for easier debugging) use \texttt{./configure --disable-nativecode}.

You can also enable the documentation generation by using
\texttt{./configure --enable-doc}. This userguide and the ocamldoc
documentation will appear in \texttt{doc/}.

\subsection{Test}

Testing the core library (should be instantaneous):
\begin{verbatim}
 $ make -C src/ndcore test
\end{verbatim}
Same test, then running {\tt bedwyr} on some examples (may take up to
one minute):
\begin{verbatim}
 $ make test
\end{verbatim}


% ============================================================================
\section{Syntax}

Although the concrete syntax of Bedwyr was originally derived from that
of \lp{} in the Teyjus implementation\cite{nadathur99cade}, it has
now evolved in such a way that,by design, it resembles that of Abella
(Andrew Gacek,
\urldef{\thisurl}\url{http://abella.cs.umn.edu/}\ahrefurl{\thisurl}), a
proof assistant itself derived both from \lp{} and from the first
implementation of Bedwyr. This means that there is no compatibility
whatsoever between versions 1.2 and 1.3.

\subsection{Tokens}

Names for objects such as types, predicates, constants and variables are
character strings built with letters, digits and the special
characters \verb.-^<>=~+*&:|., \verb.?`$'. and \verb._/@#!..
As a general rule, variables starting with an uppercase character
(\verb.A-Z.) in a query or the body of a definition are free variables
that are implicitly existentially quantified, a digit cannot be the
first character of a name, and the contiguous characters \verb./*.
cannot be part of names, even different ones.
Moreover, keywords are implicitly excluded from the present
definition.

More precisely, we divide the special characters into three categories:
\begin{itemize}
  \item infix characters (\verb.-^<>=~+*&:|.)
  \item prefix characters (\verb.?`$'.)
  \item tail characters (\verb._/@#!.)
\end{itemize}
which gives us three token categories:
\begin{itemize}
  \item upper names, starting with \verb.A-Z. and containing any letter,
    digit, or prefix or tail character (ie anything but an infix):
    \verb.Foo?0., \verb.B@r., \verb.My_Var'.
  \item prefix names, starting with \verb.a-z. or a prefix character and
    containing any letter, digit, or prefix or tail character (ie
    anything but an infix): \verb.l33t., \verb.h#sh., \verb.?Your_Var.
  \item infix names, the only ones to contain infix characters, and
    containing nothing else: \verb.-->., \verb.|=., \verb.^^.
\end{itemize}
As already said, a free variable will be denoted by an upper name. A
bound variable can use either an upper or a prefix name, but it is
customary to use upper names for explicitly existentially quantified
variables, so that visual consistency is kept when we take advantage of
the implicit quantification of free variables. A type or predicate can
use any prefix name, a constant can use any prefix or infix name.

Though no name can begin with \verb._.\footnote{Actually, such names
exist and are accepted by the parser, but are rejected by the type,
constant and predicate declarations, as they are read-only names, only
used for undocumented, internally defined predicates (usually
experimental, non-logical, and with side-effects).}, this character can
serve as a placeholder for an unspecified term, in which case it serves
as a fresh one-time free variable, except when used instead of a
variable name in a binding (abstraction or quantification), where it
serves as vacuous abstraction.

\subsection{Concrete syntax}

The grammar for formulas and terms is given in Figure~\ref{concrete}.
The abstraction over variable \verb.x. in \verb.term. is denoted by
\verb.x\term. | which is read as $\lambda x. term$. The scope of the
infix $\lambda$-abstraction extends to the right as far as possible
inside of a term, but not across formula operators: the term
\verb.x\ y\ f y x = g x y. is parenthesized as
\verb.(x\ (y\ ((f y) x))) = ((g x) y)..
On the other hand, the scope of the n-ary quantifiers extends to the
right even over formula operators: the formula
{\tt forall x y, f y x = g x y} is parenthesized as
{\tt forall x y, (((f y) x) = ((g x) y))}.

\begin{figure}
\[\begin{array}{rclp{5cm}}
form &::=& term\;\verb.=.\;term         & equality \\
     & | & form\;\verb./\.\;form        & conjunction \\
     & | & form\;\verb.\/.\;form        & disjunction \\
     & | & form\;\verb.->.\;form        & implication \\
     & | & \verb.forall x y z,.\;form   & universal quantification \\
     & | & \verb.exists X Y Z,.\;form   & existential quantification \\
     & | & \verb.nabla x y z,.\;form    & generic quantification \\
     & | & term                         & \\
term &::=& atom*                        & application \\
     & | & atom*\;id\verb.\.\;term      & application on an
                                          abstraction \\
     & | & term\;infix\;term            & (partial) infix application \\
atom &::=& true\;|\;false               & {\tt prop} \\
     & | & \verb.(.term\verb.).         & \\
     & | & \verb.(.formula\verb.).      & \\
     & | & \verb.(.infix\verb.).        & prefix version of an infix
                                          constant\\
     & | & \verb.".string\verb.".       & {\tt string} \\
     & | & [0-9]+                       & {\tt nat} \\
     & | & id                           & name \\
\end{array}\]
\caption{Grammar for formulas and terms.}
\label{concrete}
\end{figure}

In its normal state, an infix constant has to be at least of arity 2:
{\tt (x **)} raises a parsing error, but {\tt (x ** y)} is legal, and can
even be applied to another term if {\tt **} is of arity arity 3 or more.
It is also possible to use the prefix version of an infix constant by
surrounding it with parenthesis, in which case any arity is permitted:
{\tt (**) x} and {\tt (**) x y z} are both syntacticly legal.

A normal application has precedence over the application of an infix
constant: {\tt w x ** y z} is parenthesized as {\tt (w x) ** (y z)};
otherwise, all infix constants have the same priority, and are
right-associative, to mimic the behavior of {\tt ->}.

%\subsection{Abstract syntax}
%TODO say something about the distinction formula/term?


% ============================================================================
\section{User interface}

When you run Bedwyr, you specify a file or collection of files for it to
load; the objects declared and defined in those files will be loaded in
the corresponding order. You can then use the interactive toplevel to
ask queries against those definitions or run meta-commands.

\subsection{Definition files}

Definition files are usually named with a \verb|.def| extension.
You can find several of them in the \verb.examples. directory
of the Bedwyr distribution.
They contain declarations for the types (${\tt Kind}\;id\;{\tt type.}$)
and the constants (${\tt Type}\;id\;type{\tt .}$) that are not
predefined, definitions, and meta-commands.
Definitions are given as blocks with a header containing declarations
and a body containing a set of clauses, in which uppercase variables are
implicitly universally quantified:
\[\begin{array}{rcl}
  def\_block    &::=& {\tt Define}\;declarations{\tt .} \\
                & | & {\tt Define}\;declarations\;{\tt by}\;
                     definitions{\tt.} \\
  declarations  &::=& decl\;{\tt,}\;declarations \\
                & | & decl\\
  decl          &::=& flavour\;id\;{\tt :}\;type \\
  flavour       &::=& inductive\;|\;coinductive\;|\\
  definitions   &::=& clause\;{\tt;}\;definitions \\
                & | & clause\\
  clause        &::=& id\;atom*\;{\tt:=}\;formula \\
                & | & id\;atom* \\
\end{array}\]

An empty definition for a given predicate means that it is always false,
an bodiless clause means that the head is always true.
A predicate can only depend on predicates defined up to its definition
block, so multiple predicates on one block is the only way to achieve
mutual recursion.

The only meta-command that is really intended to definition files is the
include command:
\begin{verbatim}#include "another/file.def".\end{verbatim}
The \verb.#include. can really be seen as the inclusion of another file,
as Bedwyr doesn't have any namespace or module system.

\subsubsection{Emacs mode}

Assuming Bedwyr is installed in system folders, you can use the Emacs
mode for Bedwyr by adding these two lines to your \url{~/.emacs} file:
\begin{verbatim}
(load "/usr/share/bedwyr/contrib/emacs/bedwyr.el")
(setq bedwyr-program "/usr/bin/bedwyr")
;; Of course you can change both locations to wherever you want.
\end{verbatim}

Then you should be able to load any \verb:.def: file
and have syntax highlighting and some rough auto-indenting.
Also if you do \verb.C-c C-c. it will start Bedwyr
and load the current file you are working on.

\subsubsection{Vim syntax highlighting}

There is also a basic syntax highlighting file in
\verb;contrib/vim/syntax/;:
\begin{itemize}
  \item copy {\tt /usr/share/bedwyr/contrib/vim/syntax/bedwyr.vim} to
    your \url{~/.vim/syntax/} directory to make it available
  \item copy {\tt /usr/share/bedwyr/contrib/vim/ftdetect/bedwyr.vim} to
    your \url{~/.vim/ftdetect/} directory to have it used automatically
    for all {\tt *.def} files
\end{itemize}

\subsection{Toplevel}

The interactive toplevel is automatically loaded once the files have
been parsed, unless the flag \verb.-I. is passed to Bedwyr. You can
either query a formula, or run a command. In queries, free variables are
the only objects that can be used without prior declaration, and their
instantiations in solutions are displayed.

In the following example we load a set of definitions and check a
theorem about it: $\lambda x.x\;x$ has no simple type.

\begin{verbatim}
 $ src/bedwyr examples/lambda.def
[...welcome message...]
?= (exists T, wt nil (abs x\ app x x) T) -> false.
Yes.
More [y] ?
No more solutions.
?=
\end{verbatim}

The query expresses the fact that there does not exist a simple type for
type for $\lambda x (x x)$.
Notice that we had to use the term \verb+(abs x\ app x x)+ instead of
\verb+(x1\ x1 x1)+: the former encodes $\lambda x (x x)$ by mapping
object-level abstraction to {\tt abs} and object-level application to
{\tt app}, while the latter encodes this $\lambda$-term directly,
which is not possible anymore since type-checking was introduced in
version 1.3.

Most of the errors that can stop the reading of a file (parsing or
typing error, undeclared object, etc) are correctly caught by the
toplevel, though the line number is usually irrelevant.

\subsubsection{Line editing}

Bedwyr has no line editing facilities at all. Thus, we recommend using
\texttt{ledit} or \texttt{rlwrap}, which provides such features. Get one
of them from your usual package manager or at
\urldef{\thisurl}\url{http://pauillac.inria.fr/~ddr/ledit/}\ahrefurl{\thisurl}
or \urldef{\thisurl}\url{http://utopia.knoware.nl/~hlub/uck/rlwrap/}
\ahrefurl{\thisurl}.


Then you can simply run \verb.ledit src/bedwyr.. One can also define
an alias in his \verb;.bashrc;, such as the following which also
make use of \url{~/.bedwyr_history} to remember history from one session to
another:\\
\verb|alias bedwyr="ledit -h |\url{~/.bedwyr_history}%
\verb| -x /path/to/bedwyr"|.

\subsection{Session management commands}

Several commands alter the internal set of definitions of Bedwyr:
\verb.#include., \verb.#session., \verb.#reload. and \verb.#reset..
\begin{itemize}
  \item
    \verb.#include. is meant to be used in \verb;.def; files.
  \item
    \verb.#session. is a better \verb.#include. meant for query mode.
    It accepts any number of filenames as parameters, and this set of files
    will be remembered as the current \emph{session}.
    When you pass filenames on Bedwyr's command line,
    it is equivalent to call \verb.#session. with these definition files.
  \item
    \verb.#reload. clears all the definitions,
    and then reloads all the session's files. It is useful if they have
    been changed.
  \item
    \verb.#reset. clears all the definitions and empties the session.
\end{itemize}

\subsection{Assertions}

Three kinds of assertions can be used in definition files.
These tests are not executed unless the \verb.-t. flag has been passed
on Bedwyr's command-line, in which case any assertion failure is fatal.
\begin{itemize}
\item
\verb.#assert F. checks that the formula $F$ has at least one solution.
\item
\verb.#assert_not F. checks that $F$ has no solution.
\item
\verb.#assert_raise F. checks that the proof-search for $F$ triggers
a runtime error.
\end{itemize}

Our examples include a lot of assertions, to make sure that definitions have
(and keep) the intended meaning. These assertions are also the basis of
Bedwyr's correctness and performance tests ran using \verb.make test..

\subsection{Other commands}
\begin{itemize}
  \item Tabling
    \begin{itemize}
      \item \verb.#equivariant on. (un)sets an alternative tabling mode
      \item \verb.#clear_table p. clears the results cached for a
        predicate
      \item \verb.#clear_tables. clears all cached results
    \end{itemize}

  \item Output
    \begin{itemize}
      \item \verb.#debug on. adds a lot of output to the proof search
      \item \verb.#time on. displays computation times between results
      \item \verb.#env. lists all declared objects with their kind or
        type
      \item \verb.#typeof F. type-checks a formula, and also displays
        the type of its free variables
      \item \verb.#show_table p. prints the table of a predicate
      \item \verb+#show_table p "file.def"+ outputs the table of a
        predicate in a Bedwyr-compatible format
    \end{itemize}

  \item General purpose
    \begin{itemize}
      \item \verb.#help.
      \item \verb.#exit.
    \end{itemize}
\end{itemize}


% ============================================================================
\section{The logic behind Bedwyr}
\label{logic}

The logic behind Bedwyr, named LINC, is an extension to a higher-order
version of intuitionistic logic that has been developed over the past
few years.  The acronym LINC, which stands for ``lambda, induction,
nabla, and co-induction'', lists the main novel components of the
logic.  In particular, $\lambda$-terms are supported directly (and,
hence, the $\lambda$-tree syntax approach to higher-order abstract
syntax is supported \cite{miller00cl}).  Induction and co-induction
are also available.  The nabla $\nabla$ quantifier has been added to
this logic in order to increase the expressiveness of programs using
$\lambda$-tree syntax in negated situations.  The proof theory of LINC
is given in \cite{miller05tocl,tiu04phd}.

Below we provide a high-level overview of the logical aspects of Bedwyr.
More explicit information on this system can be found in
\cite{tiu05eshol}: n.b., the name ``Level 0/1'' in that paper has now
been replaced by Bedwyr.

\subsection{Built-in treatment of bindings}

Bedwyr treats $\lambda$-abstractions within terms as primitive as well
as allowing for variables of function type and quantifiers within
formulas ($\forall$, $\exists$, $\nabla$).  The system
implements ``higher-order pattern unification'' (also called
\Ll-unification) \cite{miller91jlc}.   This kind of unification
appears to be the weakest extension to first-order unification that
treats bindings as a primitive.  A number of automated deduction systems
implement this kind of unification (eg, Twelf, Teyjus, Coq, and
Minlog).  Full $\beta$-conversion is implemented by Bedwyr as well.

\begin{figure}
\begin{verbatim}
% The predicate a holds for 3, 5, and 2.
a (s (s (s z))).
a (s (s (s (s (s z))))).
a (s (s z)).

% The less-than-or-equal relation
leq z N.
leq (s N) (s M) :- leq N M.

% Compute the maximum of a
maxa N :- a N, pi x\ a x => leq x N.
\end{verbatim}
\caption{Computing the maximum of a defined predicate (\lp).}
\label{maxa-lp}
\end{figure}

\subsection{Syntax and semantics of definitions}

Some systems implementing aspects of higher-order logic programming,
such as \lp{}, accept the ``open-world assumption'': any conclusion
drawn in their logic will hold in any extension of the underlying logic
programming language.
For example, consider the \lp{} program in Figure~\ref{maxa-lp} (the
signature has been left out), where the last clause has an implication
\verb.=>. in the goal. During proof search, this implication
causes \lp{} to add \verb.c., a new eigenvariable of the proof search,
and to extend the current program with an atomic fact about it:
\verb.(a c).. In such a new world, however, the {\tt leq} relation to
does not have any information about this ``non-standard'' number
{\tt c}.
Bedwyr on the contrary accepts the ``closed-world assumption'': the
notion of programs is replaced by {\em definitions} that capture the
``if-and-only-if'' closure of logic programs. In the corresponding
(excerpt from a) program in Figure~\ref{maxa-bdw}, Bedwyr takes the
assumption \verb.(a c). and asks ``Given the assumption that
\verb.(a c). is true, how could have it been proved?'' The natural
answer to this is that that assumption could have been proved if
\verb.c. was either 3 or 5 or 2. Thus, this will cause a case analysis:
in particular, the query \verb.(maxa N). will cause the following goals
to be considered:
\begin{center}
  \tt(a N)\qquad(leq 3 N)\qquad(leq 5 N)\qquad(leq 2 N)
\end{center}
Here we use the numeric symbols `2', `3', etc., as abbreviations of the
corresponding terms formed using \texttt{z} and \texttt{s}. The usual
approach to unification and depth-first proof search will now produce
the proper maximum value. This change allows Bedwyr to give a
computational interpretation to finite failure and to do deduction that
encodes model checking.

\begin{figure}
\begin{verbatim}
Define a : int -> prop by
  a (s (s (s z)));
  a (s (s (s (s (s z)))));
  a (s (s z)).

Define leq : int -> int -> prop by
  leq z N;
  leq (s N) (s M) := leq N M.

Define maxa : int -> prop by
  maxa N := a N, forall  x, a x -> leq x N.
\end{verbatim}
\caption{Computing the maximum of a defined predicate (Bedwyr).}
\label{maxa-bdw}
\end{figure}

\bigskip
There are two orthogonal extensions to higher-order intuitionistic
logic that have been incorporated into Bedwyr.   We describe them next.

\subsection{Symmetry of finite success and finite failure}

The underlying logic of {\em fixed points} (also known as {\em
definitions})
\cite{girard92mail,schroeder-Heister93lics,mcdowell03tcs,momigliano03types}
contains an inference rule that allows for failure in unification
(and, hence, in simple proof search) to be turned into a success.
Thus, simple forms of ``negation-as-failure'' can be naturally
captured in Bedwyr and the underlying logic.  It is also possible to
describe both {\em may} and {\em must} behaviors in process calculi.
For example, not only can one code reachability in process calculus
but bisimulation is also possible.  One way to view this enhancement
to proof search is the following: Let $A$ and $B$ be two atomic
formulas.  Then, finite success is captured by proving the sequent
$\longrightarrow A$, finite failure is captured by proving the sequent
$A\longrightarrow$, and simulation is captured by proving the sequent
$A\longrightarrow B$.

\subsection{The \texorpdfstring{$\nabla$}{nabla} quantifier}

In order to treat specifications using $\lambda$-tree syntax
properly, it appears that a new quantifier, called $\nabla$, is
necessary.  If finite success is all that is needed, the $\nabla$ can
be replaced with the universal quantifier.  When finite failure is
involved, however, the $\nabla$ quantifier plays an independent role.
See \cite{miller05tocl,tiu04phd,tiu05concur} for more on this
quantifier.  It is worth pointing out that we know of no examples
involving $\nabla$ that do not also involve $\lambda$-tree syntax.


% ============================================================================
\section{How Bedwyr works}
\label{sec:howto}

A better understanding of the tool is probably needed to get your work
done.

\subsection{Proof search within LINC}
\label{psearch}

Bedwyr is a proof-search engine for a small fragment of the LINC
logic.  In principle, Bedwyr uses two provers.  {\em Prover 1} is similar to
the depth-first interpreter used in \lp{}.  The main
difference is in the proof of an implication.
To prove an implication $A\Rightarrow B$, prover 1  calls {\em prover 0}
to enumerate all possible solutions
$\{\theta_i\;|\;i=1,\ldots,n\}$ for $A$,
and then prover 1 tries to prove $B\theta_1\wedge\dots\wedge B\theta_n$.
If $A$ has no solution (that is, if $n=0$), the implication is true.
The substitutions generated by prover 1 are for existential\footnote{
We avoid the usual names (\emph{logic variables} for existential variables and
\emph{eigenvariables} for universal variables) in order to clearly separate the
high-level description given here from the implementation, which is not
detailed, but in which the class of a variable isn't static.}
variables, as usual in logic programming.
On the other hand, the substitutions generated by prover 0 are for
universal variables.

To illustrate this,
consider the following goal:
\[ \forall x \qs (\exists y \qs x=s~y) \Rightarrow x=0 \Rightarrow false \]
Bedwyr will call prover 1 on it. The prover introduces a universal variable
and reaches the first implication.
It then calls prover 0 on $(\exists y \qs x = s~y)$.
Prover 0 introduces an existential variable $y$,
and the unification instantiates $x$ to get the only solution.
Back to prover 1, we have to prove $(s~y = 0 \Rightarrow false)$
where $y$ is still an existential variable. Prover 0 is given $s~y=0$
to prove and fails to do so: that failure is a success for prover 1.

We've seen in Section \ref{logic} with the \verb.maxa. example
(Figure~\ref{maxa-bdw}) how this treatment of the implication allows
Bedwyr to check formulas which are not provable in traditional (pure) logic
programming languages such as \lp{}.
As often, this novelty has a price. The systematic enumeration leads to
infinite search for simple formulas like $(A \Rightarrow A)$ as soon as
$A$ does not have a finite number of solutions.
Further development of Bedwyr may provide
%% I wasn't sure of the claim about two implications and thought it
%% better to drop it here.  -Dale
% both kind of treatments for the implication, but more interestingly a
real support for induction and co-induction.

Prover 0 is similar to prover 1.
The first difference is this dual treatment of variables;
soundness requires another one.
Because it needs to completely destruct formulas in order to enumerate
solutions, prover 0 requires its connectives to be \emph{asynchronous} on the
left: they can be immediately destructed (introduced, in sequent
calculus terminology) without restricting provability.
This means that implication and universal quantification are forbidden on the
left of implications.

Prover 1 instantiates existential variables, and considers universal variables
as (scoped) constants. Prover 0 produces substitutions for universal variables,
considers existential variables introduced in prover 0 as constants,
but we have no satisfactory answer for existential variables introduced in
prover 1.
As a consequence, in prover 0, unification raises an run-time error
when the instantiation of an existential variable is needed.
More details about that can be found in Section \ref{restrict-logic-variables}.

\subsection{Tabling}
\label{tabling}

Proof search for a defined atom is done by unfolding the definition
for the atom, i.e. by replacing it with the body of the definition |
usually a disjunction of several clauses.
It is possible that loops occur in the proof-search,
and that the same goals arise several times.
By default, Bedwyr doesn't detect any of these issues, which makes the
proof-search much longer than needed, or infinite.
To address this, several search-directives
allow the user to instruct Bedwyr to keep records of certain
proved, disproved or ongoing formulas,
hence avoiding redundant search.
Bedwyr uses tabling in the underlying implementation to keep track of that.

Tabling is used in both prover 0 and prover 1 (see Section~\ref{psearch}).
The current implementation restricts tabling to atomic goals
with no occurrence of existential variables in prover 1.
In prover 0, only ground atomic goals (no occurrence of existential or universal
variables) are tabled.
For these goals, successes are always tabled, but some restrictions apply
to the tabling of failures, which is discussed in Section
\ref{restrict-failures-tabling}.

The use of tabling makes it possible to do proof search for
``non-terminating'' definitions, by simple loop checking.
Bedwyr will only detect loops on tabled goals, as described above.
The detection of a loop in the proof search for a predicate, say
$p\,t$, can have several interpretations, depending on whether
we consider $p\,t$ as an {\em inductive predicate} or
a {\em coinductive predicate}. In the former case, that means that
$p\,t$ is not provable, since otherwise it would contradict
the well-foundedness of inductive definitions. In the latter case,
we would have a proof of $p\,t$.

Tabling is by default not enabled in Bedwyr. To enable it, two keywords
are provided: \texttt{inductive} and \texttt{coinductive}.
To use tabling on a predicate $p$, one of them has to be added in the
declaration of $p$, in the header of the definition block.
Note that a definition block cannot contain both inductive and
co-inductive predicates at the same time, as it might lead to
contradictions | see \cite{momigliano03types} for more details.

The command \verb/#show_table pred/ allows one to inspect the contents of
\verb.pred.'s table: formulas which have been proved or disproved. The
output displays one formula per line, with the prefix \verb.P. for
proved and \verb.D. for disproved. The formulas are abstracted over by
their generic and universal variables. The relative scopings of generic
and universal variables is not displayed although that information is
present internally: such information is needed, for example, to avoid
that a proof of $(\forall x\nabla y\qs p~x~y)$ is used as a proof for
$(\nabla y\forall x\qs p~x~y)$.  The displaying of this information will
be fixed with planned extensions of the tabling mechanisms that will
implicitly allow extra axioms on $\nabla$ (see \cite{tiu06lfmtp}) in
order to be able to inspect in a meaningful way one predicate's table
from another logic program.

For example, if we define
\begin{verbatim}Define inductive neq by neq X Y := X = Y -> false.\end{verbatim}
and ask the queries \texttt{forall x, nabla y, neq x y}
and \texttt{nabla y, forall x, neq x y}, we end up with the following
puzzling table:
\begin{verbatim}
?= #show_table neq.
Table for neq contains (P=Proved, D=Disproved):
 [P] nabla x1, x2\ neq x2 x1
 [D] nabla x1, x2\ neq x2 x1
?=
\end{verbatim}
The two entries are indistinguishable by the user, but internally some
extra information does separate them.

Other tabling related commands are \verb/#clear_tables/ and
\verb/#clear_table <pred>/ which clear all or some of the tabled entries.

\subsubsection{A bisimulation example}

In some cases the table contents has important uses: for
example, once the co-inductive predicate {\tt bisim} (for bisimulation
in some of the example files) has been checked, the table for the
predicate {\tt bisim} describes a bisimulation.
We give here a simple example of checking bisimulation of finite
state automata.
The example is distributed with Bedwyr in \verb+examples/bisim.def+.
For more sophisticated examples involving the $\pi$-calculus,
we refer the reader to Section~\ref{pi-examples}.

Consider the following transition system (taken from \cite{milner99book},
page 19):
$$
\xymatrix{
   &  & p1 \ar@/_/[lld]_b \ar@/^/[dd]^a \\
p0 \ar@/_/[rru]_a \ar@/^/[rrd]_a \\
 & & p2 \ar@(ur,dr)^a \ar@/^/[llu]^b
}
\qquad
\xymatrix{
q0 \ar@/^/[rrd]^a \\
 & & q1 \ar@(ur,dr)^a \ar@/_/[dll]_b \\
q2 \ar@/_/[rru]_a
}
$$
The state $p0$ and $q0$ are bisimilar (see \cite{milner99book} for a proof).
This transition system is encoded in Bedwyr as follows:
\begin{verbatim}
Define next : state -> trans -> state -> prop by
  next p0 a p1;
  next p0 a p2;
  next p1 b p0;
  next p1 a p2;
  next p2 a p2;
  next p2 b p0;
  next q0 a q1;
  next q1 a q1;
  next q1 b q2;
  next q2 a q1.
\end{verbatim}
The bisimulation relation is encoded as the following definition
\begin{verbatim}
Define coinductive bisim : state -> state -> prop by
  bisim P Q :=
    (forall P1 A, next P A P1 ->
                     exists Q1, next Q A Q1 /\ bisim P1 Q1) /\
    (forall Q1 A, next Q A Q1 ->
                     exists P1, next P A P1 /\ bisim P1 Q1).
\end{verbatim}
Using this definition of bisimulation, Bedwyr is able to prove that
$p0$ and $q0$ are indeed bisimilar. Here is an instance of a run in Bedwyr:
\begin{verbatim}
?= bisim p0 q0.
Yes.
More [y] ? y
No more solutions.
?= #show_table bisim.
Table for bisim contains (P=Proved, D=Disproved):
 [P] (bisim p1 q1)
 [P] (bisim p2 q1)
 [P] (bisim p0 q0)
 [P] (bisim p0 q2)
?=
\end{verbatim}
The table produced gives exactly the bisimulation set
needed to prove the bisimilarity of $p0$ and $q0$, i.e.,
the set $\{(p0,q0), (p0, q2), (p1,q1), (p2,q1) \}.$

\subsubsection{Limitations of failures tabling}
\label{restrict-failures-tabling}

Successes for tabled goals are always remembered and used to avoid
re-proving them. However, storing failures is more complicated.
The failure of proof-search for some goal doesn't necessarily mean
that the formula is false: proof-search can have been aborted because
of loop detection on an inductive definition, which might actually be true.

Consider the following example.
\begin{verbatim}
Define
  inductive p,
  inductive q
by
  p := q \/ true;
  q := p.
\end{verbatim}

Let's study how the proof-search goes:
Try to prove \verb.p., unfold it, try \verb.q., unfold it, try \verb.p..
Noticing a loop, a failure happens.
However, marking \verb.q. as disproved would be wrong,
since it is clearly provable.
Indeed, the second clause for \verb.p. is tried, and yields a
success, meaning that \verb.q. is true too.

Bedwyr's implementation of tabling is very simple and conservative.
It marks goals as
disproved only if no inductive-loop failures has been involved in the failure.
This is a very strong limitation, which makes some examples (like
\verb+peterson.def+) too long because the proof-search spends a lot of time
checking again and again that the same formulas are false.

On our example, \verb.q. is not marked as disproved
but it isn't marked as proved either: the information is lost.
If \verb.p. is defined to be \verb.q.,
then it becomes false and is marked as such,
but the table for \verb.q. is left blank.

\subsection{Two runtime errors from the interpreter}

The strategy used by Bedwyr for attempting proofs is not complete.
That strategy involves using two provers (prover 0 and prover 1),
tabling, and depth-first search.
Many of the incompleteness that one encounters in
traditional logic programming languages, such as Prolog and \lp{},
resulting from depth-first search certainly reappear in Bedwyr.  We
mention two additional sources of incompleteness in the proof search
engine of Bedwyr.

\subsubsection{\texorpdfstring{\Ll{}}{Llambda} and non-\texorpdfstring{\Ll{}}{Llambda} unification problems}
A subset of \lp, called \Ll, was presented in \cite{miller91jlc} where
it was shown that an implementation of proof search could be written
in which only a small subset of higher-order unification was required.
Furthermore, that subset was decidable, unary, and did not need typing
information.  This subset of unification was called \Ll-unification in
\cite{miller91jlc} but is now more commonly referred to as {\em
  higher-order pattern unification}
\cite{nipkow93lics,nadathur05iclp}.  In that subset, variables in
functional position are applied to distinct variables which must be
bound in the scope of the binding of the functional variable.

Bedwyr allows for unrestricted applications of variables to argument
but it is only willing to solve \Ll-unification problems.  As a result,
Bedwyr will occasionally complain that it needs to solve a ``not LL
unification problem'' and stop searching for a proof.

To illustrate this aspect of Bedwyr's incompleteness, consider the
problem of specifying the instantiation of a first-order quantifier.
In particular, consider the specification
\begin{verbatim}
Kind tm, fm   type.
Type all   (tm -> fm) -> fm.
Type p         tm -> fm.
Type a         tm.
Define instan : fm -> tm -> fm -> prop by
   instan (all B) T (B T).
\end{verbatim}
Thus, {\tt instan} relates a universally quantified formula and a term
to the result of instantiating that quantifier with that term.
It is the case, however, that a
unification problem containing \verb+(B T)+ does not belong to the
\Ll{} subset.
As a result, the following query results in a runtime error.
\begin{verbatim}
?= instan (all x\ p x) a (p X).
Not LLambda unification encountered: a
?=
\end{verbatim}
In some situations, a specification can be written so that the
problematic unification is delayed to a point where the unification
problem is within the \Ll{} restriction.  In this particular case, if
the clause in the definition of {\tt instan} is rewritten to the
logically equivalent clause
\begin{verbatim}
   instan (all B) T S := S = (B T).
\end{verbatim}
this same query now returns an appropriate solution.
\begin{verbatim}
?= instan (all x\ p x) a (p X).
Solution found:
 X = a
More [y] ? 
No more solutions.
?=
\end{verbatim}
An improvement to Bedwyr would be for it to automatically delay
unification problems that are outside the \Ll-subset: delaying
``difficult'' unification problems in the hope that future
instantiations and $\beta$-reduction will make them ``simple'' is
employed in such systems as Twelf and the second version of the Teyjus
implementation of \lp{} \cite{teyjus.website}. 

\subsubsection{Restriction on the occurrences of logic variables}
\label{restrict-logic-variables}

As we have already noted, in the current implementation of Bedwyr
there are restrictions on negative occurrences of logic variables |
i.e. to the left of an implication.
This restriction arises from the fact that we do not
have a satisfactory and comprehensive understanding of unification in
the prover 1 that incorporates such variables.  As a result, Bedwyr
is incomplete since it generates a run-time error in these cases.
Consider the following two queries.
\begin{verbatim}
?= exists X, X = 42 -> false.
At line 1, character 26:
Error: logic variable on the left
?= exists X, f X = 42 -> false.
Yes.
More [y] ?
No more solutions.
?=
\end{verbatim}
The first query is certainly meaningful and is provable if there is a
term different from {\tt a}: in Bedwyr, this query generates a
run-time error since it requires dealing with an prover-1
existential variable within prover-0 unification. The second query illustrates
that some instances of prover-0 unification can tolerate the
occurrences of prover-1 existential variables.

Sometimes, one can change a specification to avoid this runtime
error.  A simple example is provided by the following two queries.
\begin{verbatim}
?= exists X, (X = 42 -> false) /\ X = 17.
At line 1, character 38:
Error: logic variable on the left
?= exists X, X = 17 /\ (X = 42 -> false).
Yes.
More [y] ? 
No more solutions.
?=
\end{verbatim}
Such reordering of goals is something a future version of Bedwyr might
attempt to do automatically.




% ============================================================================
\section{Examples of Bedwyr code}

\begin{flushright}
%% Tell me if I'm violating conventions about quotes,
%% but I thought it'd be nicer to avoid the split after "than the",
%% so I moved Mark Twain on its own line to avoid a too long second line.
%% - David
Few things are harder to put up with \\
than the annoyance of a good example. \\ -- Mark Twain
\end{flushright}

The distribution of Bedwyr comes with several examples of its use.
These examples can be classified roughly as follows.

\begin{description}
% TODO check that said examples were translated,
% remove old ones, and put them in folders according to
% the following list
\item[Basic examples] These examples are small and illustrate some
  simple aspects of the system.

\item[Model checking] Some simple model-checking-style examples are
  provided.

\item[Games] Bedwyr allows for a simple approach to explore for
  winning strategies in some simple games, such as tic-tac-toe.

\item[$\lambda$-calculus] Various relations and properties of the
  $\lambda$-calculus are developed in some definition files.

\item[Simulation and bisimulation] These relationships between
  processes where an important class of examples for which the theory
  behind Bedwyr was targeted.  Examples of checking simulation is done
  for abstract transition systems, value-passing CCS, and the
  $\pi$-calculus.  The $\pi$-calculus examples are of particular note:
  all side-conditions for defining the operational semantics and
  bisimulation are handled directly and declaratively by the logic
  underlying Bedwyr.  See Section~\ref{pi-examples} below for some more
  details about the $\pi$-calculus in Bedwyr.

\end{description}

\subsection{Hypothetical reasoning}

For those familiar with \lp{}, a key difference between
Bedwyr and \lp{} is that the latter allows for ``hypothetical''
reasoning and such reasoning is central to the way that \lp{} treats
bindings in syntax.    Bedwyr treats implication and universals in
goal formulas in a completely different way: via the closed world
assumption.

Sometimes, when dealing with $\lambda$-tree syntax in Bedwyr, one
wishes to program operations as one might do in \lp{}.  This is possible
in the sense that one can write in Bedwyr an interpreter for suitable
fragments of \lp{}.  This is done, for example, in the {\tt seq.def}
definition file.  There is a goal-directed proof search procedure for a
small part of hereditary Harrop formulas (in particular, the minimal
theory of the fragment based on $\top$, $\wedge$, $\supset$, and
$\forall$).  This interpreter is prepared to use a logic program that
is stored as a binary predicate.  For example, in \lp{}, one would write
type checking for simple types over the untyped $\lambda$-calculus
encoded using {\tt app} and {\tt abs} as
\begin{verbatim}
typeof (app M N) B :- typeof M (arrow A B), typeof N A.
typeof (abs R) (arrow A B) :- pi x\ typeof x A => typeof (R x) B.
\end{verbatim}
The hypothetical reasoning that is involved in typing the object-level
$\lambda$-binder in the second clause above is not available directly
in Bedwyr.  One can, however, rewrite these clauses as simply
``facts'' in Bedwyr:
\begin{Verbatim}
Define simple : form -> form -> prop by
  simple (type_of (app M N) Tb)
    (type_of M (Ta \mytilde> Tb) && type_of N Ta);
  simple (type_of (abs R) (Ta \mytilde> Tb))
    (for_all x\ type_of x Ta --> type_of (R x) Tb).

Define atom : form -> prop by
  atom (type_of X T).
\end{Verbatim}
See \verb+progs_small.def+ for the complete declarations. The first
definition describes a logic program called {\tt simple} that directly
encodes the above \lp{} program; the second definition tells the
interpreter in {\tt seq.def} how to recognize an object-level atomic
formula.  A call to \verb+seq atom simple tt (type_of term Ty)+ will now
attempt to perform simple type checking on {\tt term}.  Specifically, it
should now be possible to prove in Bedwyr the goal
\begin{verbatim}
(exists Ty, seq atom simple tt (type_of (abs x\ app x x) Ty))
  -> false.
\end{verbatim}
in other words, the self-application $\lambda x(x x)$ does not have a
simple type.

This ``two-level approach'' of specification uses Bedwyr as a
meta-language in which a simple intuitionistic logic is encoded as an
object logic: computations can then be specified in the object-logic
in the usual way and the Bedwyr can be used to reason about that specification.
This general approach has been described in more detail in
\cite{miller06ijcar}.

\subsection{The \texorpdfstring{$\pi$}{pi}-calculus example in more detail}
\label{pi-examples}

To illustrate another example and how it
can be used, consider the implementation of the $\pi$-calculus that is
contained in the example file \verb+pi/pi-new.def+.  Of the several
things defined in that file, the operational semantics for the
$\pi$-calculus is given using one-step transitions: for a specific
example, see Figure~\ref{one-step}.  First notice that the syntax of
clauses is almost identical to that used in \lp{}.  In
particular, the backslash is used to denote the $\lambda$-binder.  The
main syntactic difference is that the head and body of clauses are
separated from each other using the \verb+:=+ instead of the
\verb+:-+ (turnstile).  The former symbol is used to remind
the Bedwyr user that programs are used to define ``if and only if''
completions of specifications (whereas, in \lp{} the \verb+:-+ is the
more usual ``if'' interpretation).  Unlike \lp{}, Bedwyr is
untyped so there are no {\tt kind} and {\tt type} declarations and no
static checking of your source code.

Beyond the syntactic differences, the operational semantics of \lp{}
and Bedwyr differ significantly.  If a specification is simply a Horn
clause program, the two systems coincide. They differ in the operational
interpretation of implication: in Bedwyr, to prove $A\supset B$, all
possible ways to prove $A$ are explored and
for each answer substitution $\theta$ that is found, the goal
$B\theta$ is attempted (see Section~\ref{psearch}).  Bedwyr also
contains the $\nabla$-quantifier \cite{miller05tocl}.

\begin{figure}
\begin{verbatim}
[...]
onep (in X M)    (dn X) M;
one  (out X Y P) (up X Y) P;
one  (match X X P) A Q := one  P A Q;
onep (match X X P) A M := onep P A M;
one  (nu P) A (nu Q)          :=
  nabla x, one  (P x) A (Q x);
onep (nu P) A (y\ nu x\Q x y) :=
  nabla x, onep (P x) A (y\ Q x y);
one (par P Q) tau (par R T) :=
  exists X Y M, onep P (dn X) M /\
    one Q (up X Y) T /\ R = (M Y);
one (par P Q) tau (par R T) :=
  exists X Y M, onep Q (dn X) M /\
    one P (up X Y) R /\ T = (M Y);
[...]
\end{verbatim}
\caption{Some lines in {\tt pi-new.def} used to define one-step transitions.
See the example file for the full definition.}
\label{one-step}

\begin{verbatim}
Define coinductive bisim : p -> p -> prop by
  bisim P Q :=
    (forall A P1, one P A P1 ->
      exists Q1, one Q A Q1 /\ bisim P1 Q1) /\
    (forall X M, onep P (dn X) M ->
      exists N, onep Q (dn X) N /\
      forall w, bisim (M w) (N w)) /\
    (forall X M, onep P (up X) M ->
      exists N, onep Q (up X) N /\
      nabla w, bisim (M w) (N w)) /\
    (forall A Q1, one Q A Q1 ->
      exists P1, one P A P1 /\ bisim P1 Q1) /\
    (forall X N, onep Q (dn X) N ->
      exists M, onep P (dn X) M /\
      forall w, bisim (M w) (N w)) /\
    (forall X N, onep Q (up X) N ->
      exists M, onep P (up X) M /\
      nabla w, bisim (M w) (N w)).
\end{verbatim}
\caption{The definition of (open) bisimulation.}
\label{bisim}
\end{figure}

Returning to the example in Figure~\ref{one-step}, notice that two
predicates are defined: {\tt one} and {\tt onep}.  The first one
relates a process, an action, and a process.  The second one relates a
process, an abstraction of an action, and an abstraction of a
process.  The {\tt one} predicate is used to capture ``free
transitions'' and the ``$\tau$-transition'' while the second is used
to capture bounded transitions.  See \cite{tiu04fguc,tiu05concur} for
more details on this encoding strategy for the $\pi$-calculus.

Figure~\ref{bisim} provides all that is necessary to specify (open)
bisimulation for (finite) $\pi$-calculus.  The keyword {\tt
coinductive} tells the system that it will be attempting to explore a
greatest fixed point. That keyword also enables tabling, which avoids redundant
computations and accept loops as successes (see Section \ref{tabling}).
The other cases should look natural, at least
once one understands the $\lambda$-tree approach to representing syntax
and the use of the $\nabla$-quantifier.  The main thing to point out
here is that in the specification, no special side conditions need to
be added to the system: all the familiar side conditions from the
usual papers on the $\pi$-calculus are treated by the implementation
of the Bedwyr logic: the user of the system no longer needs to deal
with them explicitly but implicitly and declaratively (via quantifier
scope, $\alpha\beta\eta$-conversion, etc).

It is now possible to test some simple examples in the system.  For
example,
\begin{verbatim}
 $ src/bedwyr examples/pi/pi-new.def
[...welcome message...]
?= bisim (in a x\ in a y\ z)
     (in a x\ nu w\ in a y\ out w w z).
Yes.
More [y] ? y
No more solutions.
?= bisim (in a x\ nu y\ match x y (out c c z))
     (in a x\ z).
Yes.
More [y] ? y
No more solutions.
?= bisim (nu x\ out a x (in c y\ match x y (out c c z)))
     (nu x\ out a x (in c y\ z)).
No.
?=
\end{verbatim}
These query prove that
$a(x).a(y).0$ and $a(x).(\nu w).a(y).w!w.0$ are bisimilar,
that
$a(x).(\nu y).[x=y].c!c.0$ and $a(x).0$ are bisimilar, and that
$(\nu x).a!x.c(y).[x=y].c!c.0$ and
$(\nu x).a!x.c(y).0$ are not bisimilar.

Several other aspects of the $\pi$-calculus are explored in the examples
files of the distribution.  For example, the file \verb+pi_modal.def+
contains a specification of the modal logics for mobility described in
\cite{milner93tcs}, and the file \verb+corr-assert.def+ specifies the
checking of ``correspondence assertions'' for the $\pi$-calculus as
described in \cite{gordon03tcs}.
% TODO rehabilitate pi_abscon.def
%The file \verb+pi_abscon.def+ specifies the polyadic $\pi$-calculus
%following \cite{milner99book}.

\bibliography{userguide}

\noindent Any paper listed above that was written by one of the
authors of this guide can be found on their respective home pages.
\end{document}

% LocalWords:  runtime Bedwyr unary Bedwyr's instan Twelf Wikipedia Bedivere% LocalWords:  Alwen
% LocalWords:  hoas Pfenning Slimmer's Gforge SVN src contrib OCaml autoconf ie
% LocalWords:  toolchain gzip bzip bedwyr ocamlopt nativecode userguide Abella
% LocalWords:  ocamldoc Gacek ary forall rclp nabla nat arity syntacticly rcl
% LocalWords:  toplevel decl flavour coinductive namespace emacs ledit rlwrap
% LocalWords:  bashrc filenames equivariant un env typeof eg Coq Minlog leq neq
% LocalWords:  maxa reachability bisimulation foundedness pred scopings bisim
% LocalWords:  predicate's automata lld rru rrd ur dr llu dll bisimilar tac CCS
% LocalWords:  bisimilarity peterson declaratively progs tt onep
