% Add some mention of
%   finite failure is incorporated.  Seldom used in the definition of
%   operational semantics (mismatch in pi-calculus is an exception).
%   Meta-level properties of SOS often are related to negation.  One
%   might want to proof the no trace reveals a secret (say, in the
%   spi-calculus).  Proving (bisim P 0) means that P makes no
%   transitions.  Etc.  Properties of SOS thus, often, involve
%   negative information.

% Give author's affiliations

\documentclass{article}
\bibliographystyle{alpha}
\usepackage{url}
\usepackage[all]{xy}
\usepackage{amsmath}

\newcommand{\lp}{$\lambda$Prolog}
\newcommand{\Ll}{$L_\lambda$}
\newcommand{\qs}{\; . \;}

\title{{\Huge A User Guide to Bedwyr}
   \thanks{Support has been obtained for this work from the following
           sources: from INRIA through the ``Equipes Associ{\'e}es''
           Slimmer and from the NSF Grant CCR-0429572 that also
           includes support for Slimmer.}
}
\author{David Baelde$^1$, Andrew Gacek$^2$, Dale Miller$^1$, \\
        Gopalan Nadathur$^2$, Alwen Tiu$^3$  \\ \\
$^1$INRIA Futurs and \'Ecole Polytechnique\\
$^2$University of Minnesota\\
$^3$Australian National University and National ICT Australia
}

\begin{document}
\maketitle
%\tableofcontents
%\newpage

\section{Overview}

Some recent theoretical work in proof search has illustrated that it
is possible to combine the following two computational principles into
one computational logic.
\begin{enumerate}
\item A symmetric treatment of finite success and finite failure.
  This allows capturing both aspects of may and must behavior in
  operational semantics and mixing model checking and logic programming.

\item Direct support for $\lambda$-tree syntax, as in $\lambda$Prolog,
       via term-level $\lambda$-binders, higher-order pattern
       unification, and the $\nabla$-quantifier.
\end{enumerate}
All these features have a clean proof theory.  The combination of
these features allow, for example, specifying rather declarative
approaches to model checking syntactic expressions containing
bindings.  The Bedwyr system is intended as an implementation of these
computational logic principles.


\paragraph{Why the name Bedwyr?}
In the legend of King Arthur and the round table, several knights
shared in the search for the holy grail.  The name of one of them,
Parsifal, is used for an INRIA team associated with the ``Slimmer''
effort.  Bedwyr was another one of those knights.  Wikipedia (using
the spelling ``Bedivere'') mentions that Bedwyr appears in {\em Monty
Python and the Holy Grail} where he is ``portrayed as a master of the
extremely odd logic in the ancient times, whom occasionally blunders."
Bedwyr is a re-implementation and rethinking of an earlier system
called Level 0/1 written by Alwen Tiu and described in
\cite{tiu05eshol}. The name ``Slimmer'' is used for a jointly funded
effort between INRIA and the University of Minnesota on
``Sophisticated logic implementations for modeling and mechanical
reasoning'': Bedwyr is an initial offering from
that collaboration.
For more information, see
\url{http://www.lix.polytechnique.fr/parsifal/} and
\url{http://slimmer.gforge.inria.fr/}.

\paragraph{What is the difference between {\em hoas} and $\lambda${\em
    -tree syntax}?} 
The term ``higher-order abstract syntax'' (hoas) was originally coined
by Pfenning and Elliott in \cite{pfenning88pldi} and names the general
practice (that was common then in, say, $\lambda$Prolog
\cite{miller87slp}) of using an abstraction in a programming or
specification language to encode binders in an object-language.  Since
the choice of ``meta-language'' can vary a great deal, the term hoas
has come to mean different things to different people.  Bedwyr follows
the {\em $\lambda$-tree} approach \cite{miller00cl} to hoas: in
particular, Bedwyr's use of $\lambda$-abstraction is meant to provide
an abstract form of syntax and to not identify syntactic objects with
functional objects, which is what usually happens when hoas is used
directly within functional programming or constructive type systems.
Instead, in Bedwyr (and \lp), $\lambda$-abstractions are more
accurately thought of as being syntactic expressions with certain weak
rules of equality which make it possible for unification to be used to
analyze the structure of $\lambda$-abstractions.  

\paragraph{Is Bedwyr efficient?}
Some care has been taken to implement the novel logical principles
that appear in Bedwyr.  In particular, the system makes extensive use
of the implementation of the suspension calculus \cite{nadathur99jflp}
and other implementation ideas developed within the Teyjus
\cite{nadathur99cade} implementation of \lp\ \cite{nadathur88iclp}.
Aspects of tabled deduction have also been added to the system
\cite{ramakrishna97cav,pientka05cade}.  We have found that Bedwyr's
performance is good enough to explore a number of interesting
examples.  It is not likely, however, that the current implementation
will support large examples.  For example, the system implements the
occur-check within logic: this is, of course, necessary for sound
deduction but it does slow unification a lot.  For example, the append
program is quadratic in the size of its first argument.  There are a
number of well-known improvements to unification that make it possible
to remove many instances of the occur-check (and making append
linear).  As of this time, such an improvement has not been added to
Bedwyr.  

\paragraph{An open source effort: Can I help?}
The Bedwyr system was conceived as a prototype that could help
validate certain proof theory and proof search topic.  In the end,
this prototype has illustrated the main principles that we hoped that
it would.  It has also pointed out a number of new topics to be
explored.  If you are interested in contributing examples, features,
or performance enhancements, or if you are interested in considering
the next generation of a system like this, please let an author of
this guide know: we are looking for contributions.

\paragraph{Background assumed}
To read this guide, we shall assume that the reader is familiar with
the implementation of proof search that is found in, say, Prolog, \lp,
or Twelf.  While familiarity with various foundations-oriented papers
(particularly, \cite{mcdowell03tcs, miller05tocl,tiu04phd}) is
important for understanding fully this system, much can be learned
from studying the examples provided in the distribution.

% ============================================================================
%\newpage
\section{Get Bedwyr}
\label{sec:install}

You can get Bedwyr from Slimmer's INRIA Gforge project page:
\url{http://gforge.inria.fr/projects/slimmer}.
There you can download tarballs or get the development version using SVN
| instructions are provided on the project page.  The development of
Bedwyr is meant to be an open source project.  If you are keen
to work on the source code and/or examples, please contact
one of the ``Project Admins'' of the project (as listed at the above url).

\subsection{Distribution layout}

The Bedwyr distribution is organized as follows:

\begin{tabular}{r@{\quad}l}
  \texttt{/src}      & Source code \\
  \texttt{/doc}      &  Documentation | you're reading it \\
  \texttt{/examples} &  Examples | reading them helps
\end{tabular}

\subsection{Build}

Bedwyr's only dependency is the OCaml compiler suite.
Then, the procedure is quite simple.

\begin{verbatim}
# ./configure
# make
\end{verbatim}

You'll get the bedwyr executable in \texttt{src/bedwyr}.

By default, Bedwyr is built using the native-code compiler \texttt{ocamlopt},
since it is much faster. If you don't have it or don't want it (e.g.
for easier debugging) use \texttt{./configure --disable-native-code}.

\subsection{Test}

\begin{verbatim}
Testing the core library:
# make -C src test
More tests, including examples in Bedwyr:
# make test
\end{verbatim}

\subsection{Line editing}

Bedwyr has no line editing facilities at all. Thus, we recommend using 
\texttt{ledit}, which provides such features.
Get it from your usual package manager or at
\url{ftp://ftp.inria.fr/INRIA/cristal/Daniel.de_Rauglaudre/Tools}.

Then you can simply run \texttt{ledit src/bedwyr}. One can also define
an alias in his \verb;.bashrc;, such as the following which also
make use of \verb#~/.bedwyr_history# to remember history from one session to 
another:\\
\verb|alias bedwyr="ledit -h ~/.bedwyr_history -x /path/to/bedwyr"|
 
\subsection{Emacs mode}

Assuming Bedwyr sits in your \verb.~/bedwyr. directory,
you can use the Emacs mode for Bedwyr by adding these two lines to your
\verb,~/.emacs, file:
\begin{verbatim}
(load "~/bedwyr/contrib/bedwyr.el")
(setq bedwyr-program "~/bedwyr/src/bedwyr")
;; Of course you can change both locations to wherever you want.
\end{verbatim}

Then you should be able to load any \verb:.def: file
and have syntax highlighting and some rough auto-indenting.
Also if you do \verb.C-c C-c. it will start Bedwyr
and load the current file you are working on.

\subsection{Vim syntax highlighting}

There is a basic syntax highlighting file in \verb;contrib/bedwyr.vim;.
To use it:
\begin{itemize}
\item
Copy it to your \verb|~/.vim/syntax| directory to use it;
\item
To have it used automatically for all \verb|*.def| files, add the
following line to your \verb;~/.vimrc; file:\\
\verb|au BufRead,BufNewFile *.def set filetype=bedwyr|
\end{itemize}


% ============================================================================
%\newpage
\section{User interface}
\label{sec:interface}

The concrete syntax of formulas in Bedwyr resemble that used by
$\lambda$Prolog in, say, the Teyjus implementation
\cite{nadathur99cade}.  While both systems implement aspects of
higher-order logic programming, they are rather different systems.
The most striking difference between these systems is that
$\lambda$Prolog accepts the ``open-world assumption'': thus, any
conclusions drawn in (the logic of) $\lambda$Prolog will hold in any
extension of the underlying logic programming language.  Bedwyr on the
contrary accepts the ``closed-world assumption'': the notion of
programs are replaced by {\em definitions} that capture the
``if-and-only-if'' closure of logic programs.  This change allows
Bedwyr to give a computational interpretation to finite failure and
to do deduction that encodes model checking.

\subsection{Syntax}

The grammar for formulas and terms is given in Figure~\ref{concrete}.
The abstraction over variable \verb.x. in \verb.term. is denoted by
\verb.x\term. | which is read as $\lambda x. term$. On top of that we
build formulas: e.g. we write \verb.pi x\ x=x.  (that is
$pi\;(\lambda{}x.(=\;x\;x))$) to represent $\forall x. x=x$.  The
scope of the infix $\lambda$-abstraction extends to the right as far
as possible: the term \verb.(x\ y\ f y x)). is parenthesized as 
\verb.(x\ (y\ ((f y) x))).

\begin{figure}
\[\begin{array}{rclp{5cm}}
form &::=& form \texttt{,}  form & conjunction \\
     & | & form \texttt{;}  form & disjunction \\
     & | & form \texttt{=>} form & implication \\
     & | & \verb.pi x\.    form  & universal quantification over $x$ \\
     & | & \verb.sigma x\. form  & existential quantification over $x$ \\
     & | & \verb.nabla x\. form  & generic quantification over $x$ \\
     & | & atomic & definition \\
     & | & term \verb.=. term & equality \\
term &::=& id & identifiers are non-empty sequences of \verb.[A-Za-z0-9/_']. \\
     & | & term \; term+ & application \\
     & | & id \verb.\. term & abstraction \\
     & | & term \; infix \; term & infix operators are
             \verb.+., \verb.-., \verb.*., \verb.->. and \verb.<-.. \\
     & | & \verb.(. term \verb.). & \\
atomic &::=& id \; term* & a possibly empty application with a constant head \\
\end{array}\]
\caption{Grammar for formulas and terms..}
\label{concrete}
\end{figure}

\subsection{Running Bedwyr}

When you run Bedwyr, you specify a file or collection of files for it
to load: the clauses in those files are considered as definitions.  You
can then ask queries against those definitions.
In queries, variables starting with an uppercase character (\verb.A-Z_.)
are implicitly quantified existentially, and their instantiations in solutions
are displayed.

In the following example we load a set of definitions and check a
theorem about it: $\lambda x.x\;x$ has no simple type.  Notice that
Bedwyr is untyped and self-application is possible (as illustrated by
the second query).

\begin{verbatim}
dbaelde@poum bedwyr % src/bedwyr examples/lambda.def
[...welcome message...]
?= (sigma T\ wt nil (abs x\ app x x) T) => false.
Yes.
More [y] ? y
No more solutions.
?= pi x\ X x = x x.
Solution found:
 X = (x1\x1 x1)
More [y] ? y
No more solutions.
\end{verbatim}
The first query expresses the fact that there does not exist a simple
type for $\lambda x (x x)$.  Notice that there is difference between
the Bedwyr term \verb+(x1\x1 x1)+ and the Bedwyr term
\verb+(abs x\ app x x)+: the latter encodes $\lambda x (x x)$ by
mapping object-level abstraction to {\tt abs} and object-level
application to {\tt app} while the former directly encodes this
$\lambda$-term directly.


\subsubsection{Definition files}

Definition files are usually named with a \verb|.def| extension.
You can find several of them in the \verb.examples. directory
of the Bedwyr distribution.
Definitions are given as a set of clauses in which uppercase
variables are implicitly universally quantified:
\[ clause ::= atomic \;\verb.:=.\; term \verb|.| \]

Definition files may also contain include commands:
\begin{verbatim}#include "another/file.def".\end{verbatim}
The \verb.#include. can really be seen as the inclusion of another file,
as Bedwyr doesn't have any namespace or module system.

Finally, if the head of the clause is missing, the body will be processed as a 
query.

\subsubsection{Session management commands}

Several commands alter the internal set of definitions of Bedwyr:
\verb.#include., \verb.#session., \verb.#reload. and \verb.#reset..
\begin{itemize}
  \item
    \verb.#include. is meant to be used in \verb;.def; files.
  \item
    \verb.#session. is a better \verb.#include. meant for query mode.
    It accepts any number of filenames as parameters, and this set of files
    will be remembered as the current \emph{session}.
    When you pass filenames on Bedwyr's command line,
    it is equivalent to call \verb.#session. with these definition files.
  \item
    \verb.#reload. clears all the definitions,
    and then reloads all the session's files. It is useful if they have 
    been changed.
  \item
    \verb.#reset. clears all the definitions and empties the session.
\end{itemize}

\subsubsection{Assertions}

Three kind of assertions can be used in definition files.
These tests are not executed unless the \verb.-t. flag has been passed 
on Bedwyr's command-line, in which case any assertion failure is fatal.
\begin{itemize}
\item
\verb.#assert F. checks that the formula $F$ has at least one solution.
\item
\verb.#assert_not F. checks that $F$ has no solution.
\item
\verb.#assert_raise F. checks that the proof-search for $F$ triggers
a runtime error.
\end{itemize}

Our examples include a lot of assertions, to make sure that definitions have 
(and keep) the intended meaning. These assertions are also the basis of
Bedwyr's correctness and performance tests ran using \verb.make test..

\section{The logic behind Bedwyr}
\label{logic}

The logic behind Bedwyr is an extension to a higher-order version of
intuitionistic logic, named LINC.
It has been developed over the past few years.  The acronym, which
stands for ``lambda, induction, nabla, and co-induction'', lists the
main novel components of the logic.  In particular, $\lambda$-terms
are supported directly (and, hence, the $\lambda$-tree syntax approach
to higher-order abstract syntax is supported \cite{miller00cl}).
Induction and co-induction are also available.  The nabla $\nabla$
quantifier has been added to this logic in order to increase the
expressiveness of programs using $\lambda$-tree syntax in negated situations. 
The proof theory of LINC is given in \cite{miller05tocl,tiu04phd}.

Below we provide a high-level overview of the
logical aspects of Bedwyr that differentiate it from (the logics
underlying) \lp\ and Prolog.  More explicit information on this system
can be found in \cite{tiu05eshol}: n.b., the name ``Level 0/1'' in
that paper has now been replaced by Bedwyr.


\paragraph{Built-in treatment of bindings}
Bedwyr treats $\lambda$-abstractions within terms as primitive as well
as allowing for variables of function type and quantifiers within
formulas ($\forall$, $\exists$, $\nabla$).  The system
implements ``higher-order pattern unification'' (also called
$L_\lambda$-unification) \cite{miller91jlc}.   This kind of unification
appears to be the weakest extension to first-order unification that
treats bindings as a primitive.  A number of ``proof search'' systems
implement this kind of unification (eg, Twelf, Teyjus, Coq, and
Minlog).  Full $\beta$-conversion is implemented by Bedwyr as well.
\lp\ similarly implements all these features, although \lp\ implements
full higher-order unification (that is, it is not restricted to the
$L_\lambda$ subset).  Another different with \lp\ is that Bedwyr does
not have a built-in notion of types, while \lp\ is simply typed.

\begin{figure}
\begin{verbatim}
% The predicate a holds for 3, 5, and 2.
a (s (s (s z))).
a (s (s (s (s (s z))))).
a (s (s z)).

% The less-than-or-equal relation
leq z N.
leq (s N) (s M) := leq N M.

% Compute the maximum of a
maxa N := a N, pi x\ a x => leq x N.
\end{verbatim}
\caption{Computing the maximum of a defined predicate.}
\label{maxa}
\end{figure}

\paragraph{Syntax and semantics of clauses}
Clauses in Bedwyr resemble those in \lp, but the underlying proof
search semantics of those clauses can differ significantly.  For example,
consider the Bedwyr program in Figure~\ref{maxa}.  These clauses are
identical to those in \lp\ except for the use of \verb+:=+ instead of
\verb+:-+.  Actually the ``proof search semantics'' for all but the
last clause above is the same in Bedwyr and \lp\ (and, in this case,
Prolog too).  The last clause, however, has an implication \verb+=>+
in the goal, something also allowed in \lp.  During proof search,
however, this occurrence of the implication in \lp\ causes the current
program to be extended with an atomic fact of the form \verb+(a c)+,
where {\tt c} is a new eigenvariable of the proof search.  In reality,
\lp\ uses the ``open-world assumption'': it adds a new item {\tt c}
and a new fact about it {\tt (a c)}.  In such a new world, however, the 
{\tt leq} relation to does not have any information about this
``non-standard'' number {\tt c}.  In contrast to this, Bedwyr will
take the assumption  {\tt (a c)} and ask ``Given the assumption that 
{\tt (a c)} is true, how could have it been proved?''  The natural
answer to this is that that assumption could have been proved if {\tt
c} was either 3 or 5 or 2.  Thus, this analysis will cause a case
analysis: in particular, the query {\tt (maxa N)} will cause the
following goals to be considered:
\begin{verbatim}
        (a N)     (leq 3 N)    (leq 5 N)    (leq 2 N)
\end{verbatim}
Here we use the numeric symbols `2', `3', etc., as abbreviations of
the corresponding terms formed using \texttt{0} and \texttt{s}.
The usual approach to unification and depth-first proof search will
now produce the proper maximum value.

\bigskip
There are two orthogonal extensions to higher-order intuitionistic
logic that have been incorporated into Bedwyr.   We describe them next.

\paragraph{Symmetry of finite success and finite failure}
The underlying logic of {\em fixed points} (also known as {\em
definitions})
\cite{girard92mail,schroeder-Heister93lics,mcdowell03tcs,momigliano03types}
contains an inference rule that allows for failure in unification
(and, hence, in simple proof search) to be turned into a success.
Thus, simple forms of ``negation-as-failure'' can be naturally
captured in Bedwyr and the underlying logic.  It is also possible to
describe both {\em may} and {\em must} behaviors in process calculi.
For example, not only can one code reachability in process calculus
but bisimulation is also possible.  One way to view this enhancement
to proof search is the following: Let $A$ and $B$ be two atomic
formulas.  Then, finite success is captured by proving the sequent
$\longrightarrow A$, finite failure is captured by proving the sequent
$A\longrightarrow$, and simulation is captured by proving the sequent
$A\longrightarrow B$.

\paragraph{The $\nabla$ quantifier}
In order to treat specifications using $\lambda$-tree syntax
properly, it appears that a new quantifier, called $\nabla$, is
necessary.  If finite success is all that is needed, the $\nabla$ can
be replaced with the universal quantifier.  When finite failure is
involved, however, the $\nabla$ quantifier plays an independent role.
See \cite{miller05tocl,tiu04phd,tiu05concur} for more on this
quantifier.  It is worth pointing out that we know of no examples
involving $\nabla$ that do not also involve $\lambda$-tree syntax.


\section{How to use Bedwyr}
\label{sec:howto}

A better understanding of the tool is probably needed to get your work
done. 

\subsection{Proof-search within LINC}
\label{psearch}

Bedwyr is a proof search engine for a small fragment of the LINC
logic.  In principle, Bedwyr uses two provers.  {\em Prover 1} is similar to
a depth-first interpreter used in $\lambda$Prolog.  The main
difference is in the proof of an implication.  
To prove an implication $A\Rightarrow B$, prover 1  calls {\em prover 0}
to enumerate all possible solutions
$\{\theta_i\;|\;i=1,\ldots,n\}$ for $A$, 
and then prover 1 tries to prove $B\theta_1\wedge\dots\wedge B\theta_N$.
If $A$ has no solution (that is, if $n=0$), the implication is true.
The substitutions generated by prover 1 are for existential\footnote{
We avoid the usual names (\emph{logic variables} for existential variables and 
\emph{eigenvariables} for universal variables) in order to clearly separate the 
high-level description given here from the implementation, which is not 
detailed, but in which the class of a variable isn't static.}
variables, as usual in logic programmming.
On the other hand, the substitutions generated by prover 0 are for
universal variables. 

To illustrate this, let's study a simple example.
Consider the following goal:
\[ \forall x \qs (\exists y \qs x=s~y) \Rightarrow x=0 \Rightarrow false \]
Bedwyr will call prover 1 on it. The prover introduces an universal variable
and reaches the first implication.
It then calls prover 0 on $(\exists y \qs x = s~y)$.
Prover 0 introduces an existential variable $y$,
and the unification instantiates $x$ to get the only solution.
Back to prover 1, we have to prove $(s~y = 0 \Rightarrow false)$.
where $y$ is still an existential variable. Prover 0 fails on the 
left-hand side, that's a success for prover 1.

We've seen in Section \ref{logic} with the \verb.maxa. example
(Figure \ref{maxa}) how this treatment of the implication allows
Bedwyr to check formulas which are not provable in traditional (pure) logic 
programming languages such as \lp.
As often, this novelty has a price. The systematic enumeration leads to 
infinite search for simple formulas like $(A \Rightarrow A)$ as soon as
$A$ does not have a finite number of solutions.
Further development of Bedwyr may provide both kind of treatments for 
the implication, but more interestingly a real support for induction.

Prover 0 is very similar to prover 1.
The first difference is this dual treatment of variables;
soundness requires another one.
Because it needs to completely destruct formulas in order to enumerate 
solutions, prover 0 requires its connectives to be \emph{asynchronous} on the 
left: they can be immediately destructed without restricting provability.
This means that implication and universal quantification are forbidden on the 
left of implications.

Prover 1 instantiates existential variables, and considers universal variables 
as (scoped) constants. Prover 0 produces substitutions for universal variables,
considers existential variables introduced in prover 0 as constants,
but we have no satisfactory answer for existential variables introduced in 
prover 1.
As a consequence, in prover 0, unification raises an run-time error
when the instantiation of an existential variable is needed.
More details about that can be found in Section \ref{restrict-logic-variables}.

\subsection{Tabling}

Proof search for a defined atom is done by unfolding the definition 
for the atom, i.e., by replacing it with the body of the definition clause 
whose head matches the atom. It is possible that loops can occur in 
the proof-search, and Bedwyr does not detect them automatically. 
There is also the possibility that Bedwyr proves a predicate several
times, since it does not record whether a predicate has been proved
previously.
To address these issues, Bedwyr includes several search-directives
that the user can use to instruct Bedwyr to keep records of certain
predicates that have been proved or disproved, hence avoiding redundant search.
Bedwyr uses tabling in the underlying implementation to keep track
of proved or disproved predicates.

The use of tabling makes it possible to do proof search for
``non-terminating'' definitions, by simple loop checking.
The detection of a loop in the proof search for a predicate, say
$p\,t$, can have several interpretations, depending on whether
we consider $p\,t$ as an {\em inductive predicate} or
a {\em coinductive predicate}. In the former case, that means that
$p\,t$ is not provable, since otherwise it would contradict
the well-foundedness of inductive definitions. In the latter case,
we would have a proof of $p\,t$. Bedwyr keeps both previously proved goals 
(e.g., coinductive $p\,t$ in the previous example), 
or disproved goals (e.g., inductive $p\,t$). 

Tabling is by default not enabled in Bedwyr. Bedwyr provides two
keywords: \texttt{inductive} and \texttt{coinductive}, to explicitly
enable tabling for a given predicate. 
To use tabling on a predicate $p$, to be treated as an inductive predicate
(likewise, coinductive predicate), {\em every} definition clause for $p$
must be preceded by the keyword \texttt{inductive} 
(likewise, \texttt{coinductive}).
Note that a predicate cannot be designated as both inductive and co-inductive
at the same time, as it will lead to contradiction (see 
\cite{momigliano03types} for more details) in some cases. 

Tabling is used in both prover 0 and prover 1 (see Section~\ref{psearch}).
In the current implementation of Bedwyr, only ground atomic predicates, i.e.,
those predicates which do not contain any occurrences of logic variables, 
are tabled. 
Loops on tabled goals yield successes for coinductive definitions,
and failures for inductive ones.
Successes for tabled goals are remembered.
Failures are remembered, if no loop failure was involved.

The command \verb/#show_table pred./ allows one to inspect the
contents of \verb.pred.'s table: formulas which have been proved or
disproved.  In some cases this information has important uses: for
example, once the co-inductive predicate {\tt bisim} (for bisimulation
in some of the example files) has been checked, the table for the
predicate {\tt bisim} describes a bisimulation.  
We give here a simple example of checking bisimulation of finite
state automata. For more sophisticated examples involving the $\pi$-calculus,
we refer the reader to Section~\ref{pi-examples}.

Consider the following transition system (taken from \cite{milner99book}, 
page 19):
$$
\xymatrix{
   &  & p1 \ar@/_/[lld]_b \ar@/^/[dd]^a \\
p0 \ar@/_/[rru]_a \ar@/^/[rrd]_a \\
 & & p2 \ar@(ur,dr)^a \ar@/^/[llu]^b
}
\qquad
\xymatrix{
q0 \ar@/^/[rrd]^a \\
 & & q1 \ar@(ur,dr)^a \ar@/_/[dll]_b \\
q2 \ar@/_/[rru]_a
}
$$
The state $p0$ and $q0$ are bisimilar (see \cite{milner99book} for a proof).
This transition system is encoded in Bedwyr as follows:
\begin{verbatim}
next p0 a p1.
next p0 a p2.
next p1 b p0.
next p1 a p2.
next p2 a p2.
next p2 b p0.
next q0 a q1.
next q1 a q1.
next q1 b q2.
next q2 a q1.
\end{verbatim}
The bisimulation relation is encoded as the following definition
\begin{verbatim}
coinductive bisim P Q := 
  (pi P1\ pi A\ next P A P1 => 
                   sigma Q1\ next Q A Q1 & bisim P1 Q1) &
  (pi Q1\ pi A\ next Q A Q1 => 
                   sigma P1\ next P A P1 & bisim P1 Q1).
\end{verbatim}
Using this definition of bisimulation, Bedwyr is able to prove that
$p0$ and $q0$ are indeed bisimilar. Here is an instance of a run in Bedwyr:
\begin{verbatim}
> bisim p0 q0.
Yes.
More [y] ? y
No more solutions.
> #show_table bisim.
Table for bisim contains (P=Proven, D=Disproven):
 [P] (bisim p1 q1)
 [P] (bisim p2 q1)
 [P] (bisim p0 q0)
 [P] (bisim p0 q2)
> 
\end{verbatim}
The table produced gives exactly the bisimulation set
needed to prove the bisimilarity of $p0$ and $q0$, i.e.,
the set $\{(p0,q0), (p0, q2), (p1,q1), (p2,q1) \}.$

%% For example,
%% \begin{verbatim}
%% > bedwyr examples/pi/pi_nat.def
%% [..welcome message..]
%% ?= weak_bisim (nu u\ nu d\ (par (nat zz u)
%%                            (par (nat (s zz) d)
%%                                 (add u d t))))
%%               (nat (s zz) t).
%% Yes.
%% More [y] ? n
%% Search stopped.
%% ?= #show_table weak_bisim.
%% Table for weak_bisim contains (P=Proven, D=Disproven):
%%  [P] (x1\ weak_bisim [..] [..])
%%  [D] (weak_bisim [..] [..])
%% [..]
%% \end{verbatim}
%% In this list, formulas are marked as proved or disproved. The
%% abstractions at the head stand for universal quantifications. Generic
%% (``nabla'') variables are not abstracted, and represented by
%% \verb.n1., \verb.n2., etc.


\subsection{Two runtime errors from the interpreter}

The strategy used by Bedwyr for attempting proofs is not complete.
That strategy involves using two provers (``prover 0'' and the
``prover 1'' prover), tabling, and depth-first search.  
Many of the incompleteness that one encounters in
traditional logic programming languages, such as Prolog and \lp,
resulting from depth-first search certainly reappear in Bedwyr.  We
mention two additional sources of incompleteness in the proof search
engine of Bedwyr.

\subsubsection{\Ll\ and non-\Ll\ unification problems}
A subset of intuitionistic logic, called \Ll, was presented in
\cite{miller91jlc} and shown to have a proof search engine that needed
only a subset of higher-order unification and that that subset was
decidable, unary, and did not need typing information.  This subset
of unification was called \Ll-unification in \cite{miller91jlc} and
is also known as {\em higher-order pattern unification}
\cite{nipkow93lics,nadathur05iclp}.  In that subset, variables in
functional position are applied to distinct variables which must be
bound in the scope of the binding of the functional variable.

Bedwyr allows for unrestricted applications of variables to argument
but it is only willing to solve \Ll-unification problems.  As a result,
Bedwyr will occasionally complain that it needs to solve a ``not LL
unification problem'' and stop searching for a proof.  Since Bedwyr is
formally untyped, we use the term \Ll-unification since referring to
unification with the adjective ``higher-order'' is problematic since
the notion of order is usually attached to types.

To illustrate this aspect of Bedwyr's incompleteness, consider the
problem of specifying an input transition for, say, the value-passing
CCS, as presented in the example \verb+ccs_vp/ccs_vp.def+.  It would
be natural to represent input prefix using a $\lambda$-abstracted
process.  One step (early) transition could then be specified as the
definition class
\begin{verbatim}
trans (in C P) (i C V) (P V).
\end{verbatim}
Here, {\tt P} is denotes an abstraction over processes and
\verb+(P V)+ denotes the substitution of the inputted value {\tt V}
into the abstraction {\tt P} (Bedwyr will do $\beta$-reduction in order
to simplify this application).  It is the case, however, that a
unification problem containing \verb+(P V)+ does not below to \Ll.
As a result, the following query results in a runtime error.
\begin{verbatim}
?= trans (in a (w\ out a w zero)) A Cont.
Not LLambda unification encountered: H4
?= 
\end{verbatim}
In some situations, a specification can be written so that the
problematic unification is delay to a point that the unification
problem is within \Ll\ restriction.  In this particular case, if the clause
in the definition of {\tt trans} is rewritten to the logically
equivalent clause
\begin{verbatim}
trans (in C P) (i C V) Q := Q = (P V).
\end{verbatim}
this same query now returns an appropriate solution.
\begin{verbatim}
?= trans (in a (w\ out a w zero)) A Cont.
Solution found:
 A = (i a H9)
 Cont = (out a H9 zero)
More [y] ? y
No more solutions.
?= 
\end{verbatim}
An improvement to Bedwyr would be for it to delay unification problems
that are outside the \Ll-subset: delaying ``difficult'' unification
problems in the hope that future instantiations and $\beta$-reduction
will make them ``simple'' is employed in such systems as \lp\ and
Twelf.

\subsubsection{Restriction on the occurrences of logic variables}
\label{restrict-logic-variables}

As we have already noted, in the current implementation of Bedwyr
there are restrictions on negative occurences of logic variables |
i.e. to the left of an implication.
This restriction arises more from the fact that we do not
have a satisfactory and comprehensive understanding of unification in
the prover 1 that incorporates such variables.  As a result, Bedwyr
is incomplete since it generates a run-time error in these cases.
Consider the following two queries.
\begin{verbatim}
?= sigma x\ (x = a => false).
Error: logic variable on the left
?= sigma x\ (f x = a  => false).
Yes.
More [y] ? y
No more solutions.
?=
\end{verbatim}
The first query is certainly meaningful and is provable if there is a
term different from {\tt a}: in Bedwyr, this query generates a
run-time error since it requires dealing with an prover-1
existential variable within prover-0 unification. The second query illustrates
that some instances of unification tolerates the negative 
occurrences of logic variables.

Sometimes, one can change a specification to avoid this runtime
error.  A simple example is provided by the following two queries.
\begin{verbatim}
?= sigma x\ ((x = a  => false), x = b).
Error: logic variable on the left
?= sigma x\ ( x = b, (x = a  => false)).
Yes.
More [y] ? y
No more solutions.
?= 
\end{verbatim}
Such reordering of goals is something a future version of Bedwyr might
attempt to do automatically.

\section{Examples of Bedwyr code}

\begin{flushright}
Few things are harder to put up with than the\\
annoyance of a good example.   -- Mark Twain
\end{flushright}

The distribution of Bedwyr comes with several examples of its use.
These examples can be classified roughly as follows.

\begin{description}
\item[Basic examples] These examples are small and illustrate some
  simple aspects of the system.

\item[Model checking] Some simple model-checking-style examples are
  provided. 

\item[Games] Bedwyr allows for a simple approach to explore for
  winning strategies in some simple games, such as tic-tac-toe.

\item[$\lambda$-calculus] Various relations and properties of the
  $\lambda$-calculus are developed in some definition files.

\item[Simulation and bisimulation] These relationships between
  processes where an important class of examples for which the theory
  behind Bedwyr was targeted.  Examples of checking simulation is done
  for abstract transition systems, value-passing CCS, and the
  $\pi$-calculus.  The $\pi$-calculus examples are of particular note:
  all side-conditions for defining the operational semantics and
  bisimulation are handled directly and declaratively by the logic
  underlying Bedwyr.   Modal logic for the $\pi$-calculus is also
  illustrated by an example definition file.

\end{description}

\subsection{Hypothetical reasoning}

For those familiar with \lp, a key difference between
Bedwyr and \lp\ is that the latter allows for ``hypothetical''
reasoning and such reasoning is central to the way that \lp\ treats
bindings in syntax.    Bedwyr treats implication and universals in
goal formulas in a completely different way: via the closed world
assumption.  

Sometimes, when dealing with $\lambda$-tree syntax in Bedwyr, one
wishes to program operations as one might do in \lp.  This is possible
in the sense that one can write in Bedwyr an interpreter for suitable
fragments of \lp.  This is done, for example, in the {\tt seq.def}
definition file.  There a goal-directed proof search procedure for a
small part of hereditary Harrop formulas (in particular, the minimal
theory of the fragment based on $\top$, $\wedge$, $\supset$, and
$\forall$).  This interpreter is prepared to use a logic program that
is stored as a binary predicate.  For example, in \lp, one would write
type checking for simple types over the untyped $\lambda$-calculus
encoded using {\tt abs} and {\tt app} as
\begin{verbatim}
typeof (app M N) B :- typeof M (arrow A B), typeof N A.
typeof (abs R) (arrow A B) :- pi x\ typeof x A => typeof (R x) B.
\end{verbatim}
The hypothetical reasoning that is involved in typing the object-level
$\lambda$-binder in the second clause above is not available directly
in Bedwyr.  One can, however, rewrite these clauses as simply
``facts'' in Bedwyr
\begin{verbatim}
simple (typeof (app M N) B) 
       (and (typeof M (arrow A B)) (typeof N A)).
simple (typeof (abs R) (arrow A B)) 
       (forall x\ (typeof x A) -> (typeof (R x) B)).
atom (typeof A B).
\end{verbatim}
The first two atomic formulas describe a logic program called {\tt
simple} that directly encodes the above \lp\ program: the third fact
above tells the interpreter in {\tt seq.def} how to recognize an
object-level atomic formula.  A call to
\verb+seq simple tt (typeof Term Type)+ will now attempt to perform
simple type checking on {\tt Term}.  Specifically, it should now be
possible to prove in Bedwyr the goal
\begin{verbatim}
(sigma T\ (seq simple tt (abs x\ app x x) T) => false).
\end{verbatim}
in other words, the self-application $\lambda x(x x)$ does not have a
simple type.

Notice that this ``two-level approach'' in which Bedwyr is used as a
meta-logic and various computations are encoded in an object-logic
(for example, type checking via minimal logic) is proposed in
\cite{miller06ijcar} as a general means of reasoning about operational
semantics.

\subsection{The $\pi$-calculus example in more detail}
\label{pi-examples}

To illustrate another example and how it
can be used, consider the implementation of the $\pi$-calculus that is
contained in the example file \verb+pi/pi.def+.  Of the several
things defined in that file, the operational semantics for the
$\pi$-calculus is given using one-step transitions: for a specific
example, see Figure~\ref{one-step}.  First notice that the syntax of
clauses is almost identical to that used in $\lambda$Prolog.  In
particular, the backslash is used to denote the $\lambda$-binder.  The
main syntactic difference is that the head and body of clauses are
separated from either other using the \verb+:=+ instead of the
\verb+:-+ (reverse turnstile).  The former symbol is used to remind
the Bedwyr user that programs are used to define ``if and only if''
completions of specifications (whereas, in \lp\ the \verb+:-+ is the
more usual ``if'' interpretation).  Unlike \lp, Bedwyr is
untyped so there are no {\tt kind} and {\tt type} declarations and no
static checking of your source code.  

Beyond the syntactic differences, the operational semantics of \lp\
and Bedwyr differ significantly.  If a specification is simply a Horn
clause program, the two systems coincide. They differ in the operational
interpretation of implication: in Bedwyr, to prove $A\supset B$, all 
possible ways to prove $A$ are explored and 
for each answer substitution $\theta$ that is found, the goal
$B\theta$ is attempted (see Section~\ref{psearch}).  Bedwyr also
contains the $\nabla$-quantifier \cite{miller05tocl}. 

\begin{figure}
\begin{verbatim}
onep (in X M)    (dn X) M.
one  (out X Y P) (up X Y) P.
one  (match X X P) A Q := one P A Q.
onep (match X X P) A M := onep P A M.
one  (nu x\P x) A (nu x\Q x) := nabla x\ one  (P x) A (Q x).
onep (nu x\P x) A (y\ nu x\Q x y) := 
                           nabla x\ onep (P x) A (y\ Q x y).
one (par P Q) tau (par R T) :=
  sigma X\ sigma Y\ sigma M\ onep P (dn X) M &
                             one Q (up X Y) T & R = (M Y).
one (par P Q) tau (par R T) :=
  sigma X\ sigma Y\ sigma M\ onep Q (dn X) M &
                             one P (up X Y) R & T = (M Y).
\end{verbatim}
\caption{Some lines in {\tt pi.def} used to define one-step
  transitions.  See the example file for the full definition.}
\label{one-step}

\begin{verbatim}
coinductive bisim P Q :=
   (pi A\ pi P1\ one P A P1 => sigma Q1\ one Q A Q1 &
                                         bisim P1 Q1) &
   (pi X\ pi M\  onep P (dn X) M => sigma N\ onep Q (dn X) N & 
                                     pi w\ bisim (M w) (N w)) &
   (pi X\ pi M\ onep P (up X) M => sigma N\ onep Q (up X) N &
                                     nabla w\ bisim (M w) (N w)) &
   (pi A\ pi Q1\ one Q A Q1 => sigma P1\ one P A P1 & 
                                         bisim Q1 P1) &
   (pi X\ pi N\  onep Q (dn X) N => sigma M\ onep P (dn X) M & 
                                     pi w\ bisim (N w) (M w)) &
   (pi X\ pi N\ onep Q (up X) N => sigma M\ onep P (up X) M &
                                     nabla w\ bisim (N w) (M w)).
\end{verbatim}
\caption{The definition of (open) bisimulation.}
\label{bisim}
\end{figure}

Returning to the example in Figure~\ref{one-step}, notice that two
predicates are defined: {\tt one} and {\tt onep}.  The first one
relates a processes, an action, and a process.  The first relates a 
processes, an abstractions of an action, and an abstraction of a
process.  The {\tt one} predicate is used to capture ``free
transitions'' and the ``$\tau$-transition'' while the second is used
to capture bounded transitions.  See \cite{tiu04fguc,tiu05concur} for
more details on this encoding strategy for the $\pi$-calculus.

Figure~\ref{bisim} provides all that is necessary to specify (open)
bisimulation for (finite) $\pi$-calculus.  The keyword {\tt
coinductive} tells the system that it will be attempting to explore a
greatest fixed point.  The other cases should look natural, at least
once one understand the $\lambda$-tree approach to representing syntax
and the use of the $\nabla$-quantifier.  The main thing to point out
here is that in the specification, no special side conditions need to
be added to the system: all the familiar side conditions from the
usual papers on the $\pi$-calculus are treated by the implementation
of the Bedwyr logic: the user of the system no longer needs to deal
with them explicitly but implicitly and declaratively (via quantifier
scope, $\alpha\beta\eta$-conversion, etc).

It is now possible to test some simple examples in the system.  For
example,
\begin{verbatim}
$ src/bedwyr examples/pi/pi.def
Bedwyr welcomes you....

?= bisim (in a x\ in a y\ z) (in a x\ nu w\ in a y\ out w w z).
Yes.
More [y] ? y
No more solutions.
?= bisim (in a x\ nu y\ match x y\ out c c z) (in a x\ z).
Yes.
More [y] ? y
No more solutions.
?= bisim (nu x\ out a x (in c y\ match x y (out c c z)))
                (nu x\ out a x (in c y\ z)).
No.
?= 
\end{verbatim}
These query prove that 
$a(x).a(y).0$ and $a(x).(\nu w).a(y).w!w.0$ are bisimilar, 
that
$a(x).(\nu y).[x=y].c!c.0$ and $a(x).0$ are bisimilar, and that 
$(\nu x).a!x.c(y).[x=y].c!c.0$ and 
$(\nu x).a!x.c(y).0$ are not bisimilar.

% Link's to Dale's bibref file.  After Dale runs the aux2bib (see Makefile),
% then one can use the local userguide.bib file.
%\bibliography{../../../../../pap/references/master}

\bibliography{userguide}

\end{document}
