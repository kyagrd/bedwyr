\part{Tutorials}
\label{tutorials}


% ============================================================================
% ============================================================================
\section{Released examples}

\begin{flushright}
Few things are harder to put up with \\
than the annoyance of a good example. \\ -- Mark Twain
\end{flushright}

The distribution of Bedwyr comes with several examples of its use.
These examples can be classified roughly as follows.

\begin{description}
  % TODO check that said examples were translated,
  % remove old ones, and put them in folders according to
  % the following list, and add examples of use for files under each
  % item, including items with BC/FC
  \item[Basic examples] These examples are small and illustrate some
    simple aspects of the system.

  \item[Model checking] Some simple model-checking-style examples are
    provided.

  \item[Games] Bedwyr allows for a simple approach to explore for
    winning strategies in some simple games, such as tic-tac-toe.

  \item[\lc{}] Various relations and properties of the
    \lc{} are developed in some definition files.

  \item[Simulation and bisimulation] These relationships between
    processes where an important class of examples for which the theory
    behind Bedwyr was targeted.  Examples of checking simulation is done
    for abstract transition systems, value-passing CCS, and the
    \pc{}.  The \pc{} examples are of particular note:
    all side-conditions for defining the operational semantics and
    bisimulation are handled directly and declaratively by the logic
    underlying Bedwyr.  See \autoref{pi-examples} below for some more
    details about the \pc{} in Bedwyr.
\end{description}


% ============================================================================
% ============================================================================
\section{Hypothetical reasoning}

For those familiar with \lp{}, a key difference between Bedwyr and \lp{}
is that the latter allows for ``hypothetical'' reasoning and such
reasoning is central to the way that \lp{} treats bindings in syntax.
Bedwyr treats implication and universals in goal formulas in a
completely different way: via the closed world assumption.

Sometimes, when dealing with $\lambda$-tree syntax in Bedwyr, one
wishes to program operations as one might do in \lp{}.  This is possible
in the sense that one can write in Bedwyr an interpreter for suitable
fragments of \lp{}.  This is done, for example, in the \path{seq.def}
definition file.  There is a goal-directed proof search procedure for a
small part of hereditary Harrop formulas (in particular, the minimal
theory of the fragment based on $\top$, $\land$, $\supset$, and
$\forall$).  This interpreter is prepared to use a logic program that
is stored as a binary predicate.  For example, in \lp{}, one would write
type checking for simple types over the untyped \lc{}
encoded using \lstinline{app} and \lstinline{abs} as in \autoref{lp:typeof}.
The hypothetical reasoning that is involved in typing the object-level
$\lambda$-binder in the second clause above is not available directly
in Bedwyr.  One can, however, rewrite these clauses as simply
``facts'' in Bedwyr (\autoref{bd:typeof}).

\begin{figure}
  \centering
  \begin{lstlisting}[%
    style=lprolog,caption={Simple typing in Teyjus.},label={lp:typeof}]
typeof (app M N) B :- typeof M (arrow A B), typeof N A.
typeof (abs R) (arrow A B) :- pi x\ typeof x A => typeof (R x) B.
  \end{lstlisting}
\end{figure}
\begin{figure}
  \centering
  \lstinputlisting[%
    firstline=12,lastline=19,%
    caption={Simple typing in Bedwyr (from
    \protect\path{examples/progs_small.def}).},%
    label=bd:typeof%
  ]{include/linc/typeof.def}
\end{figure}

The first definition describes a logic program called \lstinline{simple}
that directly encodes the above \lp{} program; the second definition
tells the interpreter in \lstinline{seq.def} how to recognize an
object-level atomic formula.  A call to
\lstinline{seq atom simple tt (type_of Term Ty)} will now attempt to
perform simple type checking on \lstinline{Term}.  Specifically, it is
possible to prove in Bedwyr the goal
\begin{center}\begin{lstlisting}
(exists Ty, seq atom simple tt (type_of (abs x\ app x x) Ty))
  -> false.
\end{lstlisting}\end{center}
or, in other words, that the self-application $\lambda x(x x)$ does not
have a simple type.

This ``two-level approach'' of specification uses Bedwyr as a
meta-language in which a simple intuitionistic logic is encoded as an
object logic: computations can then be specified in the object-logic in
the usual way and then Bedwyr can be used to reason about that
specification.  This general approach has been described in more detail
in \cite{miller06ijcar,gacek.twolevel}.


% ============================================================================
% ============================================================================
\section{The \texorpdfstring{\pc{}}{pi-calculus} example in more detail}
\label{pi-examples}

To illustrate another example and how it can be used, consider the
implementation of the \pc{} that is contained in the example file
\path{pi/pi.def}.  Of the several things defined in that file, the
operational semantics for the \pc{} is given using one-step transitions:
for a specific example, see \autoref{bd:pi-one-step}.

Beyond the syntactic differences, the operational semantics of \lp{} and
Bedwyr differ significantly.  If a specification is simply a Horn clause
program, the two systems coincide.  They differ in the operational
interpretation of implication: in Bedwyr, to prove $A\supset B$, all
possible ways to prove $A$ are explored and for each answer substitution
$\theta$ that is found, the goal $B\theta$ is attempted (see
\autoref{psearch}).  Bedwyr also contains the $\nabla$-quantifier
\cite{miller05tocl}.

\begin{figure}
  \centering
  \lstinputlisting[%
    linerange={117-121,124-124,127-127,130-131},%
    caption={Some one-step transitions (from
    \protect\path{examples/pi/pi.def}).},%
    label=bd:pi-one-step%
  ]{include/linc/pi.def}
\end{figure}

\begin{figure}
  \centering
  \lstinputlisting[%
    firstline=193,lastline=210,%
    caption={(Open) bisimulation (from
    \protect\path{examples/pi/pi.def}).},%
    label=bd:pi-bisim%
  ]{include/linc/pi.def}
\end{figure}

Returning to the example in \autoref{bd:pi-one-step}, notice that two
predicates are defined: \lstinline{one} and \lstinline{onep}.  The first
one relates a process, an action, and a process.  The second one relates
a process, an abstraction of an action, and an abstraction of a process.
The \lstinline{one} predicate is used to capture ``free transitions''
and the ``$\tau$-transition'' while the second is used to capture
bounded transitions.  See \cite{tiu04fguc,tiu05concur} for more details
on this encoding strategy for the \pc{}.

\autoref{bd:pi-bisim} provides all that is necessary to specify (open)
bisimulation for (finite) \pc{}.  The keyword \lstinline{coinductive}
tells the system that it will be attempting to explore a greatest fixed
point.  That keyword also enables tabling, which avoids redundant
computations and accept loops as successes (see \autoref{tabling}).  The
other cases should look natural, at least once one understands the
$\lambda$-tree approach to representing syntax and the use of the
$\nabla$-quantifier.  The main thing to point out here is that in the
specification, no special side conditions need to be added to the
system: all the familiar side conditions from the usual papers on the
\pc{} are treated by the implementation of the Bedwyr logic: the user of
the system no longer needs to deal with them explicitly but implicitly
and declaratively (via quantifier scope, $\alpha\beta\eta$-conversion,
etc.).

It is now possible to test some simple examples in the system, for
example \autoref{bd:pi-run}.
\begin{figure}
  \centering
  \begin{lstlisting}[%
    style=bedwyr-repl,%
    caption={Run on \protect\path{examples/pi/pi.def}},%
    label=bd:pi-run]
?= bisim (in a x\ in a y\ z)
     (in a x\ nu w\ in a y\ out w w z).
Yes.
More [y] ? y
No more solutions.
?= bisim (in a x\ nu y\ match x y (out c c z))
     (in a x\ z).
Yes.
More [y] ? y
No more solutions.
?= bisim (nu x\ out a x (in c y\ match x y (out c c z)))
     (nu x\ out a x (in c y\ z)).
No.
?=
  \end{lstlisting}
\end{figure}
These query prove that $a(x).a(y).0$ and $a(x).(\nu w).a(y).w!w.0$ are
bisimilar, that $a(x).(\nu y).[x=y].c!c.0$ and $a(x).0$ are bisimilar,
and that $(\nu x).a!x.c(y).[x=y].c!c.0$ and $(\nu x).a!x.c(y).0$ are not
bisimilar.

Several other aspects of the \pc{} are explored in the examples files of
the distribution.  For example, the file \path{pi/pi_modal.def} contains
a specification of the modal logics for mobility described in
\cite{milner93tcs}, the file \path{pi/corr-assert.def} specifies the
checking of ``correspondence assertions'' for the \pc{} as described in
\cite{gordon03tcs}, and the file \path{pi/pi_abscon.def} specifies the
polyadic \pc{} following \cite{milner99book}.
