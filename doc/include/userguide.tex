\part{A User's Guide to Bedwyr}


% ============================================================================
\section{Overview}

Some recent theoretical work in proof search has illustrated that it
is possible to combine the following two computational principles into
one computational logic:
\begin{enumerate}
  \item a symmetric treatment of finite success and finite failure.
    This allows capturing both aspects of may and must behavior in
    operational semantics and mixing model checking and logic programming.

  \item direct support for $\lambda$-tree syntax, as in \lp{},
    via term-level $\lambda$-binders, higher-order pattern
    unification, and the $\nabla$-quantifier.
\end{enumerate}
All these features have a clean proof theory.  The combination of
these features allow, for example, specifying rather declarative
approaches to model checking syntactic expressions containing
bindings.  The Bedwyr system is intended as an implementation of these
computational logic principles.

\paragraph{Why the name Bedwyr?}
In the legend of King Arthur and the round table, several knights
shared in the search for the holy grail.  The name of one of them,
Parsifal, is used for an INRIA team where Bedwyr is currently
developed. Bedwyr was another one of those knights.  Wikipedia (using
the spelling ``Bedivere'') mentions that Bedwyr appears in {\em Monty
Python and the Holy Grail} where he is ``portrayed as a master of the
extremely odd logic in the ancient times, whom occasionally blunders."
Bedwyr is a re-implementation and rethinking of an earlier system
called Level 0/1 written by Alwen Tiu and described in
\cite{tiu05eshol}. It was an initial offering from ``Slimmer'', a
jointly funded effort between INRIA and the University of Minnesota on
``Sophisticated logic implementations for modeling and mechanical
reasoning'' from 2005 to 2010. For more information, see
\url{http://slimmer.gforge.inria.fr/}.

\paragraph{What is the difference between {\em hoas} and $\lambda${\em
    -tree syntax}?}
The term ``higher-order abstract syntax'' (hoas) was originally coined
by Pfenning and Elliott in \cite{pfenning88pldi} and names the general
practice (that was common then in, say, \lp{}
\cite{miller87slp}) of using an abstraction in a programming or
specification language to encode binders in an object-language.  Since
the choice of ``meta-language'' can vary a great deal, the term
``hoas'' has come to mean different things to different people.  When
hoas is used directly within functional programming or constructive
type systems, syntax with bindings contains functional objects, which
make rich syntactic manipulations difficult.  Bedwyr, on the other
hand, follows the {\em $\lambda$-tree} approach \cite{miller00cl} to
hoas: in particular, Bedwyr's use of $\lambda$-abstraction is meant to
provide an abstract form of syntax in which only the names of bindings
are hidden: the rest of the structure of syntactic expressions
remains.

\paragraph{Is Bedwyr efficient?}
Some care has been taken to implement the novel logical principles
that appear in Bedwyr.  In particular, the system makes extensive use
of the implementation of the suspension calculus \cite{nadathur99jflp}
and other implementation ideas developed within the Teyjus
\cite{nadathur99cade} implementation of \lp{} \cite{nadathur88iclp}.
Aspects of tabled deduction have also been added to the system
\cite{ramakrishna97cav,pientka05cade}.  We have found that Bedwyr's
performance is good enough to explore a number of interesting
examples.  It is not likely, however, that the current implementation
will support large examples.  For example, the system implements the
occur-check within logic: this is, of course, necessary for sound
deduction but it does slow unification a lot.  For example, the append
program is quadratic in the size of its first argument.  There are a
number of well-known improvements to unification that make it possible
to remove many instances of the occur-check (and making append
linear).  As of this time, such an improvement has not been added to
Bedwyr.

\paragraph{An open source effort: Can I help?}
The Bedwyr system was conceived as a prototype that could help
validate certain proof theory and proof search topic.  In the end,
this prototype has illustrated the main principles that we hoped that
it would.  It has also pointed out a number of new topics to be
explored.  If you are interested in contributing examples, features,
or performance enhancements, or if you are interested in considering
the next generation of a system like this, please let an author of
this guide know: we are looking for contributions.

\paragraph{Background assumed}
To read this guide, we shall assume that the reader is familiar with
the implementation of proof search that is found in, say, Prolog,
\lp{}, or Twelf.  While familiarity with various foundations-oriented
papers (particularly,
\cite{mcdowell03tcs,miller05tocl,tiu04phd,baelde08lfmtp,tiu10tocl}) is
important for understanding fully this system, much can be learned
from studying the examples provided in the distribution.


% ============================================================================
\section{Getting Bedwyr}

Different means of getting Bedwyr are listed on Slimmer's INRIA Gforge
project site:
\url{http://slimmer.gforge.inria.fr/bedwyr/#download}.
You can either download tarballs, get any development version using SVN,
or use Slimmer's unofficial Apt repository | instructions are provided
on the project page.
The development of Bedwyr is meant to be an open source project.
If you are keen to work on the source code and/or examples, please contact
one of the ``Project Admins'' of the project (as listed at
\url{https://gforge.inria.fr/projects/slimmer/}.

\subsection{Distribution layout}

The Bedwyr distribution is organized as follows:

\begin{tabular}{r@{\quad}l}
  \texttt{src/}      & Source code \\
  \texttt{doc/}      & Documentation | you're reading it \\
  \texttt{contrib/}  & Emacs and Vim support \\
  \texttt{examples/} & Examples | reading them helps
\end{tabular}

\subsection{Build}

Bedwyr's main build dependency is the OCaml compiler suite.
You also need some standard tools you may already have, especially
autoconf and GNU make (part of the GNU toolchain),
and bash, tar, gzip and bzip2 for the installation.

Then, the procedure is quite simple.

\begin{verbatim}
 $ autoconf
 $ ./configure
 $ make
\end{verbatim}

You'll get the bedwyr executable in \texttt{src/bedwyr}.

By default, Bedwyr is built using the native-code compiler \texttt{ocamlopt},
since it is much faster. If you don't have it or don't want it (e.g.
for easier debugging) use \texttt{./configure --disable-nativecode}.

You can also enable the documentation generation by using
\texttt{./configure --enable-doc} and \texttt{make doc}. This userguide
and the ocamldoc documentation will be generated in \texttt{doc/}.

\subsection{Test}

Testing the core library (should be instantaneous):
\begin{verbatim}
 $ make -C src/ndcore test
\end{verbatim}
Same test, then running {\tt bedwyr} on some examples (may take up to
one minute):
\begin{verbatim}
 $ make test
\end{verbatim}


% ============================================================================
\section{User interface}

When you run Bedwyr, you specify a file or collection of files for it to
load; the objects declared and defined in those files will be loaded in
the corresponding order. You can then use the interactive toplevel to
ask queries against those definitions, or call meta-commands.
Those queries and commands can also be specified on the command-line via
the option \verb.-e. (e.g.
\texttt{bedwyr -e 'X = 0.' -e '\#typeof X = 1.'}), in which case they are
processed in the order they are given, after the files and before the
toplevel.

\subsection{Definition files}

Definition files are usually named with a \verb|.def| extension. You can
find several of them in the \verb.examples. directory of the Bedwyr
distribution. They contain definitions, meta-commands, and declarations
for the types (${\tt Kind}\;id\;{\tt type.}$) and the constants (${\tt
Type}\;id\;type{\tt .}$) that are not predefined. Definitions are given
as blocks with a header containing declarations and an optional body
containing a set of clauses, in which uppercase variables are implicitly
universally quantified:
\[\begin{array}{rcl}
  def\_block    &::=& {\tt Define}\;declarations{\tt .} \\
                & | & {\tt Define}\;declarations\;{\tt by}\;
                     definitions{\tt.} \\
  declarations  &::=& decl\;{\tt,}\;declarations \\
                & | & decl\\
  decl          &::=& flavour\;id\;{\tt :}\;type \\
  flavour       &::=& inductive\;|\;coinductive\;|\\
  definitions   &::=& clause\;{\tt;}\;definitions \\
                & | & clause\\
  clause        &::=& id\;atom*\;{\tt:=}\;formula \\
                & | & id\;atom* \\
\end{array}\]

A predicate with an empty definition is always false;
the head of a bodiless clause is always true.
A predicate can only depend on predicates defined up to its definition
block, so multiple predicates in one block is the only way to achieve
mutual recursion.

The only meta-command that is really intended for definition files is the
include command:
\begin{verbatim}#include "another/file.def".\end{verbatim}
The \verb.#include. can really be seen as the inclusion of another file,
as Bedwyr doesn't have any namespace or module system.

\subsubsection{Emacs mode}

Assuming Bedwyr is installed in standard Linux system folders, you can
use the Emacs mode for Bedwyr by adding these two lines to your
\url{~/.emacs} file:
\begin{verbatim}
(load "/usr/share/bedwyr/contrib/emacs/bedwyr.el")
(setq bedwyr-program "/usr/bin/bedwyr")
;; Of course you can change both locations to wherever you want.
\end{verbatim}

Then you should be able to load any \verb:.def: file
and have syntax highlighting and some rough auto-indenting.
Also if you do \verb.C-c C-c. it will start Bedwyr
and load the current file you are working on.

\subsubsection{Vim syntax highlighting}

There is also a basic syntax highlighting file for vim. With a standard
system installation, the files
\url{/usr/share/vim/vimfiles/[ftdetect|syntax]/bedwyr.vim} should
suffice; otherwise do the following:
\begin{itemize}
  \item copy {\tt contrib/vim/syntax/bedwyr.vim} to
    your \url{~/.vim/syntax/} directory to make it available
  \item copy {\tt contrib/vim/ftdetect/bedwyr.vim} to
    your \url{~/.vim/ftdetect/} directory to have it used automatically
    for all {\tt *.def} files
\end{itemize}

\subsection{Toplevel}

The interactive toplevel is automatically loaded once the files have
been parsed, unless the flag \verb.-I. is passed to Bedwyr. You can
either query a formula, or run a command. In queries, free and bound
variables are
the only objects that can be used without prior declaration, and their
instantiations in solutions are displayed.

In the following example we load a set of definitions and prove that
the untyped $\lambda$-term $\lambda x.x\;x$ has no simple type.

\begin{verbatim}
 $ src/bedwyr examples/lambda.def
[...welcome message...]
?= (exists T, wt nil (abs x\ app x x) T) -> false.
Yes.
More [y] ?
No more solutions.
?=
\end{verbatim}
Notice that we had to use the term \verb+(abs x\ app x x)+ instead of
\verb+(x1\ x1 x1)+: the former encodes the untyped $\lambda$-term
$\lambda x (x x)$ by mapping
object-level abstraction to {\tt abs} and object-level application to
{\tt app}, while the latter is not a legal term in Bedwyr.  (Prior to
version 1.3, Bedwyr did not use simple typing on its own terms.)

Most of the errors that can stop the reading of a file (parsing or
typing error, undeclared object, etc) are correctly caught by the
toplevel, though the line number reported for the error is often not
useful.

\subsubsection{Line editing}

Bedwyr has no line editing facilities at all. Thus, we recommend using
\texttt{ledit} or \texttt{rlwrap}, which provides such features. Get one
of them from your usual package manager or at
\url{http://pauillac.inria.fr/~ddr/ledit/} or
\url{http://utopia.knoware.nl/~hlub/uck/rlwrap/}.

Then you can simply run \verb.ledit src/bedwyr.. One can also define
an alias in his \url{~/.bashrc}, such as the following which also
makes use of \url{~/.bedwyr_history} to remember history from one session to
another:\\
\verb|alias bedwyr="ledit -h |\url{~/.bedwyr_history}%
\verb| -x /path/to/bedwyr"|.

\subsection{Meta-commands}

\subsubsection{Session management}

Those commands alter the internal set of definitions of Bedwyr:
\begin{itemize}
  \item
    \verb.#include. is meant to be used in \verb;.def; files.
  \item
    \verb.#session. is a better \verb.#include. meant for query mode.
    It accepts any number of filenames as parameters, and this set of files
    will be remembered as the current \emph{session}.
    When you pass filenames on Bedwyr's command line,
    it is equivalent to call \verb.#session. with these definition files.
  \item
    \verb.#reload. clears all the definitions,
    and then reloads all the session's files. It is useful if they have
    been changed.
  \item
    \verb.#reset. clears all the definitions and empties the session.
\end{itemize}

\subsubsection{Assertions}

Three kinds of assertions can be used in definition files.
These tests are not executed unless the \verb.-t. flag has been passed
on Bedwyr's command-line, in which case any assertion failure is fatal.
\begin{itemize}
\item
\verb.#assert F. checks that the formula $F$ has at least one solution.
\item
\verb.#assert_not F. checks that $F$ has no solution.
\item
\verb.#assert_raise F. checks that the proof-search for $F$ triggers
a runtime error.
\end{itemize}

Our examples include a lot of assertions, to make sure that definitions have
(and keep) the intended meaning. These assertions are also the basis of
Bedwyr's correctness and performance tests ran using \verb.make test..

\subsubsection{Other commands}
\begin{itemize}
  \item Tabling
    \begin{itemize}
      \item \verb.#equivariant on. sets an alternative tabling mode
      \item \verb.#clear_table p. clears the results cached for a
        predicate
      \item \verb.#clear_tables. clears all cached results
    \end{itemize}

  \item Output
    \begin{itemize}
      \item \verb.#debug on. adds a lot of output to the proof search
      \item \verb.#time on. displays computation times between results
      \item \verb.#env. lists all declared objects with their kind or
        type
      \item \verb.#typeof F. type-checks a formula, and also displays
        the type of its free variables
      \item \verb.#show_table p. prints the table of a predicate
      \item \verb+#save_table p "file.def"+ outputs the table of a
        predicate in a Bedwyr-compatible format
    \end{itemize}

  \item General purpose
    \begin{itemize}
      \item \verb.#help.
      \item \verb.#exit.
    \end{itemize}
\end{itemize}
