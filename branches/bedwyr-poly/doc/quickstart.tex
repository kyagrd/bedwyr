\documentclass[a4paper]{article} % defaults to letter

% xypic is simply a wrapper for xy,
% but is preferred since it is understood by hevea
\usepackage[all]{xypic}

% we want Type 1 fonts instead of CM/Type 3 (vectorial vs bitmap),
% plus math fonts
\usepackage{ae,aecompl}
\usepackage{amsfonts}

% all sorts of links: internal, external, multi-line...
\usepackage[
% linkbordercolor={1 0 0}, % default: red
% citebordercolor={0 1 0}, % default: green
% filebordercolor={1 0 1}, % default: magenta
% urlbordercolor={0 1 1},  % default: cyan
  pdftitle={Bedwyr QuickStart 1.4}
]{hyperref}
\usepackage{breakurl}

% hevea commands, to be ignored by latex
\usepackage{hevea}
\loadcssfile{http://slimmer.gforge.inria.fr/bedwyr/bd.css}

\bibliographystyle{alpha}

% draw a nice tilde in the Verbatim environment
\usepackage{fancyvrb}
\newcommand{\mytilde}{\raise.3ex\hbox{$\scriptstyle\sim$}}
\RecustomVerbatimEnvironment{Verbatim}{Verbatim}{commandchars=\\\{\}}

\newcommand{\lp}{$\lambda$Prolog}
\newcommand{\Ll}{$L_\lambda$}
\newcommand{\qs}{\; . \;}

\title{{\Huge A Quick-Start Guide to Bedwyr v1.4}}
\author{Quentin Heath\\INRIA Saclay and LIX/\'Ecole Polytechnique}

\begin{document}
\maketitle

\begin{abstract}
intro (what-who-where-how, quick start)
\end{abstract}

% ============================================================================
\section{Overview}

% ============================================================================
\section{Input format}

% ============================================================================
\section{Sample files}

% ============================================================================
\section{REPL demo}

\end{document}
